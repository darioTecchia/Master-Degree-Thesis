\documentclass[14pt]{extreport}

\usepackage[utf8]{inputenc}
\usepackage[italian,english]{babel}
\usepackage{graphicx}
% \usepackage{cite}
\usepackage{amsmath}
\usepackage[table,xcdraw]{xcolor}
\usepackage[italian]{minitoc}% per gli accenti
\usepackage{fancybox}
\usepackage{verbatim}
\usepackage{setspace}
\usepackage{url}
\usepackage{color}
\definecolor{lightgray}{rgb}{.9,.9,.9}
\definecolor{darkgray}{rgb}{.4,.4,.4}
\definecolor{purple}{rgb}{0.65, 0.12, 0.82}
\usepackage{subcaption}
\usepackage{listings}
\usepackage{array}
\lstset{
  breaklines=true,
  postbreak=\mbox{\textcolor{red}{$\hookrightarrow$}\space},
}
\makeatletter
\def\@tempA#1#2\@end{%
    \@tempA@{#1}%
    \ifx\relax#2\relax
    \else
        \@tempA#2\@end
    \fi
}
\def\@tempA@#1{{\noexpand#1}{{\char`\noexpand#1 \allowbreak}}1 }
\edef\@tempB{\noexpand\lstdefinelanguage{logfile}{%
    columns=fixed,%
    keepspaces=true,%
    breaklines=true,%
    literate=\@tempA 0123456789ABCDEFGHIJKLMNOPQRSTUVWXYZabcdefghijklmnopqrstuvwxyz!()*+,-./:;<=>?@[]_|`^"'\&\$\\\~\#\%\{\}\@end
}}
\@tempB
\makeatother
\usepackage{makeidx}
\usepackage{amsfonts}
\usepackage{quiver}
\usepackage{comment}
\usepackage{hyperref}
\usepackage{biblatex} %Imports biblatex package
\addbibresource{bibliography.bib} %Import the bibliography file
%se avete problemi con il testo che deve andare subito sotto una tabella allora importate il pacchetto:
%\usepackage{placeins} % e scrivete \FloatBarrier subito dopo all' \end{table}
\lstdefinelanguage{JavaScript}{
  keywords={typeof, new, true, false, catch, function, return, null, catch, switch, var, if, in, while, do, else, case, break, const, async, await},
  keywordstyle=\color{blue}\bfseries,
  ndkeywords={class, export, boolean, throw, implements, import, this, static},
  ndkeywordstyle=\color{darkgray}\bfseries,
  identifierstyle=\color{black},
  sensitive=false,
  comment=[l]{//},
  morecomment=[s]{/*}{*/},
  commentstyle=\color{purple}\ttfamily,
  stringstyle=\color{red}\ttfamily,
  morestring=[b]',
  morestring=[b]"
}

\lstset{
   language=JavaScript,
   backgroundcolor=\color{lightgray},
   extendedchars=true,
   basicstyle=\footnotesize\ttfamily,
   showstringspaces=false,
   showspaces=false,
   numbers=left,
   numberstyle=\footnotesize,
   numbersep=9pt,
   tabsize=2,
   breaklines=true,
   showtabs=false,
   captionpos=b
}
%rimuove bordi dai link
\hypersetup{
    colorlinks=true,
    linkcolor=black,
    filecolor=magenta,      
    urlcolor=cyan,
}
%apice
\usepackage{fancyhdr}
\pagestyle{fancy}
\fancyhf{}
\setlength{\headheight}{17pt}
\rhead{Cap.\thechapter}
\fancyhead[L]{\rightmark}
%pedice
\cfoot{\thepage}

\lstset{frame=tb,
	language=Java,
	numbers=left,
	keywordstyle=\color{blue},
	alsoletter={.}
}
\graphicspath{ {./Figure/} }

%formato didascalie immagini
\DeclareCaptionFormat{custom}
{%
    \textbf{\small #1#2}\textit{\small #3}
}
\captionsetup{format=custom}

%formato capitoli, paragrafi e sottoparagrafi
% \usepackage{titlesec}
% \titleformat{\chapter}{\normalfont\huge\bf}{\thechapter.}{20pt}{\huge\bf}
% \titleformat{\section}{\normalfont}{\thesection.}{18pt}{\normalfont\bf}
% \titleformat{\subsection}{\normalfont}{\thesubsection.}{16pt}{\normalfont\bf}

\newcommand{\quotes}[1]{``#1''}
\newcommand{\foo}{\hspace{-3.7pt}$\bullet$ \hspace{5pt}}

%%%%%%%%%%%%%%%%%%%%%%%%%%%%
\begin{document}
\onehalfspace
\selectlanguage{italian}
\begin{titlepage}
  \begin{center}
    \begin{figure}
        \includegraphics[width=3cm, height=3cm]{unisa.png}
        \centering
      \end{figure}
    {\Large Università degli Studi di Salerno}\\[0.2truecm]
    {\large Dipartimento di Informatica}\\
    \hrulefill
    \vfill
    {\large Corso di Laurea Magistrale in Informatica}\\[0.2truecm]
    %{\Large Informatica}\\
    \vfill\vfill
    {\LARGE
      {\bf 
        Blockchain nell'era quantistica, una soluzione per la Proof-of-Stake
      }
    }
    
    \vfill\vfill
    
    
    {\bf Docente} \hfill {\bf Candidato} \\
    Prof.ssa \textbf{Genoveffa Tortora} \hfill  \textbf{Tecchia Dario}
    \centerline{\hfill 0522500736}
    
    
    \vfill
    \hrulefill 
    \begin{center} Anno Accademico 2021-2022 \end{center}
    
  \end{center}
\end{titlepage}
  
\setcounter{page}{1} 		
\pagenumbering{roman}
\newpage
%%%%%%%%%%%%%%%%%%%%%%%%%%%%
% Ringraziamenti
\chapter*{Ringraziamenti}
Non vorrei ringraziare chi c'è o chi ci sarà, ma chi ci è stato.
%%%%%%%%%%%%%%%%%%%%%%%%%%%%
% Abstract
\chapter*{Abstract}
Nonostante la Blockchain sia considerata una della tecnologie più all'avanguardia dell'ultimo decennio, a minacciare la sua sicurezza è la crescita esponenziale che sta avendo un altro tipo di tecnologia: l'informatica quantistica. In particolare, due algoritmi quantistici, l'algoritmo di fattorizzazione di Shor e l'algoritmo di Grover, che possono risolvere alcuni problemi computazionali in tempi considerevolmente minori rispetto alle controparti non quantistiche. In questo lavoro di tesi proponiamo un aggiornamento all'attuale Proof-of-Stake in cui andiamo a sostituire l'attuale schema di firma con uno quantum-safe, lo SPHINCS. Rendendo così la PoS sicura ad attacchi quantistici.
%%%%%%%%%%%%%%%%%%%%%%%%%%%%
% ToC
\pagenumbering{roman}
\tableofcontents
\listoffigures %elenco figure
\listoftables %elenco tabelle
%%%%%%%%%%%%%%%%%%%%%%%%%%%%
\newpage
\pagenumbering{arabic}
%Introduction
\chapter*{Introduzione}
\addcontentsline{toc}{chapter}{\protect\numberline{}Introduzione}
Possiamo sicuramente affermare che dalla pubblicazione del white paper \textit{"Bitcoin: A Peer-to-Peer Eletronic Cash System"}, pubblicato nel 2008 da Satoshi Nakamoto, il mondo digitale sta subendo un drastico cambiamento. In molti hanno iniziato ad investire in questa nuove monete digitali figlie di questa nuova tecnologia, chiamata \textbf{Blockchain}. Oltre ad avver dato vita a questa nuova forma monetaria, la Blockchain si sta spingendo verso nuovi orizzonti d'utilizzo, come ad esempio la pubblica amministrazione e la sanità. Una delle principali caratteristiche della Blockchain è la mancanza di un'entità centrale.

Ma ad oggi sta emergendo una nuova tecnologia che va a compromettere i punti di forza della blockchain, il \textbf{quantum computing}. Quest'ultima tecnologia è senza dubbio in rapida crescita, passando da pochi qbit (l'unità di informazione quantistica) a centinaia di qbit in soli cinque anni, e si prevede un'ulteriore crescita fino al 2023 toccando numeri a quattro cifre. Tutta questa capacità computazionale data dai computer quantistici va a minacciare la robustezza degli attuali algoritmi di firma digitale che sono alla base della blockchain. Quindi bisogna trovare al più presto una soluzione che renda gli attuali schemi di firma digitale resistenti agli attacchi quantistici.

È stata proposta una variante di \textbf{Proof-of-Stake} a cui viene rimpiazzato l'attuale algoritmo di cifratura con uno schema quantum safe, lo \textit{SPHINCS}. Inoltre, è stata presentata un'implementazione del sistema proposto, \textbf{Quantum Resistant Chain} (o \textbf{QRChain}).

Infine, negli sviluppi futuri, è stato proposto un aggiornamento di Ethereum in cui si sostituisce l'attuale algoritmo di consenso, la \textbf{Proof-of-Work}, con la più sicura Proof-of-Stake predisponendo quindi Ethereum ad accogliere QRChain nel proprio core. Oltre all'aggiornamento di Ethereum è stato proposto l'aggiornamento dello schema di cifratura SPHINCS con il più recente \textit{SPHINCS+} che porta notevoli miglioramenti sia dal punto di vista delle performance che della sicurezza.

L'elaborato si articola su cinque capitoli:
\begin{itemize}
  \item Nel \textbf{Capitolo 1} vengono affrontati nello specifico i fondamenti della computazione quantistica. Dopodichè si forniscono i metodi di gestione di quest'ultima e come avviene la codifica dell'informazione reale all'interno di un sistema quantistico.
  \item Nel \textbf{Capitolo 2} viene illustrata la storia della Blockchain, partendo dai Sistemi di Chaum fino ad arrivare al Bitcoin. Viene poi illustrata l'architettura e il suo funzionamento, i diversi tipi di algoritmi di consenso e le generazioni di blockchain fino ad ora.
  \item Nel \textbf{Capitolo 3} vengono illustrate le principali criticità della Proof-of-Stake, i modelli di attacco all'algoritmo e come queste debolezze possono essere mitigate grazie ad accorgimenti architetturali e la sostituzione degli schemi di firma digitale. Viene infine presentato lo SPHINCS e i suoi punti di forza.
  \item Nel \textbf{Capitolo 4} viene presentata QRChain, il relativo meccanismo di consenso, il meccanismo di selezione della catena e l'implementazione di quest'ultima. QRChain è l'implementazione di una blockchain avente come algoritmo di consenso la Proof-of-Stake e come schema di crifratura lo SPHINCS.
  \item Nel \textbf{Capitolo 5} viene proposto un aggiornamento ad Ethereum basato su Proof-of-Stake e metodi di cifratura quantum safe, chiamata Ethereum 2.0, e il successore dello SPHINCS avente diverse migliorie computazionali, chiamato SPHINCS+. 
\end{itemize}

%%%%%%%%%%%%%%%%%%%%%%%%%%%%
% Chapters
\chapter{Quantum Computing}
L'informatica quantistica combina l'informatica tradizionale con la meccanica quantistica ed è un campo di ricerca in rapida crescita. Questo interesse verso il calcolo quantistico inizia negli anni settanta con lo sviluppo di una serie di tecniche per ottenere il controllo completo di singoli sistemi quantistici. 

Fino a quel momento, la teoria classica dell'informatica era stata fondata sulla tesi, ampiamente accettata, di \textit{Church-Turing}, secondo la quale era possibile teorizzare una macchina ideale, nota come \textit{Macchina di Turing}, capace di simulare in modo efficiente qualsiasi modello di calcolo esistente.

Tuttavia, l'emergente paradigma di calcolo basato sulle proprietà meccaniche quantistiche della natura portò molti scienziati a realizzare che, mentre un computer ordinario poteva essere usato per simulare un computer quantistico, era impossibile eseguire questa simulazione in modo efficiente: ogni tentativo di simulare l'evoluzione di un generico sistema fisico-quantistico su una macchina di Turing sembrava richiedere un overhead esponenziale di risorse.

R. P. Feynman fu tra i primi fisici ad occuparsi della questione, dando le linee guida sul possibile utilizzo di sistemi quantistici come costituenti di un nuovo tipo di calcolatore; sottolineò, inoltre, come un calcolatore di questo tipo sarebbe allo stesso tempo un "simulatore" ideale per i sistemi quantistici. A partire dalle osservazioni sviluppate in quel periodo, si iniziò a costruire una nuova teoria dell'informazione, che tenesse conto delle possibilità, ancora teoriche, offerte dal calcolatore quantistico. In particolare, una nuova classificazione della complessità computazionale si rese necessaria, grazie alle peculiarità ed ai vantaggi offerti dal nuovo paradigma computazionale.

Contributi fondamentali sono stati dati da David Deutsch che, nel 1985, si chiese se le leggi della fisica quantistica potessero essere usate per derivare una versione ancora più forte della tesi di Church-Turing e tentò di definire un dispositivo computazionale che fosse capace di simulare in modo efficiente un sistema fisico arbitrario \cite{deutsch1985quantum}. Questo dispositivo sarebbe diventato la moderna concezione di un computer quantistico e che questi dispositivi potssero avere poteri di calcolo ben superiori a quelli dei computer tradizionali, indipendentemente dai loro progressi ottenibili nel calcolo classico.

Negli anni seguenti, lo studio degli algoritmi quantistici si è evoluto come un sotto-campo dell'informatica quantistica con applicazioni di diverso tipo: ricerca e ottimizzazione, machine learning, simulazione di sistemi quantistici e crittografia.

Quest'ultimo campo è quello che più ci interessa, infatti, nel 1994 Peter Shor pubblica l'algoritmo che porta il suo nome per la fattorizzazione degli interi in tempo polinomiale \cite{breaking-rsa}. Questo è stato una svolta epocale nella materia, perché un importante metodo di crittografia asimmetrica noto come RSA si basa sulla supposizione che la fattorizzazione degli interi sia difficile dal punto di vista computazionale. L'esistenza dell'algoritmo quantistico in tempo polinomiale può dimostrare che uno dei protocolli crittografici più usati al mondo sarebbe vulnerabile a un computer quantistico.

\section{Il Quantum Bit}
\subsection{Definizione di quantum bit}
L'informazione non può essere considerata separatamente dalla sua natura fisica: non si può, cioè, mantenere, modificare o trasmettere informazione senza un adeguato supporto fisico. Nei computer tradizionali viene utilizzato come modello fondamentale il \textit{bit}, che rappresenta un sistema a due stati, 0 e 1. La scelta della rappresentazione binaria è dettata dalla semplicità e comodità di realizzazione nei sistemi elettronici. Il bit classico, quindi, mantiene correttamente l'informazione relativa ad una scelta esclusiva tra i due stati possibili in cui si può trovare (con \textit{n} bit possiamo rappresentare \(2^n\) stati).

La computazione quantistica introduce una nuova unità fondamentale che prende il nome di \textit{quantum bit}, chiamato anche \textit{qubit}. Un qubit usa i fenomeni meccanici quantistici della sovrapposizione per ottenere una combinazione lineare di due stati. Un bit binario classico può rappresentare solo un singolo valore binario, ad esempio 0 o 1, ovvero può trovarsi solo in uno di due stati possibili. Un qubit tuttavia può rappresentare uno 0, un 1 o qualsiasi proporzione di 0 e 1 nella sovrapposizione di entrambi gli stati, con una determinata probabilità che si tratti di uno 0 e una determinata probabilità che si tratti di un 1.

Fisicamente viene rappresentato con un sistema microscopico a due livelli come lo spin di una particella\footnote{Lo spin è una forma di momento angolare, avendo di tale entità fisica le dimensioni e, pur non esistendo una grandezza corrispondente in meccanica classica, per analogia richiama la rotazione della particella intorno al proprio asse (viene anche definito come momento angolare intrinseco).}, la polarizzazione di un singolo fotone o due stati di un atomo ottenibili cambiando il livello energetico di un suo elettrone.

Se volessimo descriverlo matematicamente potremmo definirlo come un vettore unitario descritto in uno spazio vettoriale di Hilbert complesso bidimensionale (\(\mathbb{C}^2\)).

Per rappresentare gli elementi di uno spazio vettoriale complesso è conveniente utilizzare \textbf{la notazione di Dirac} (notazione standard della meccanica quantistica). Tale scelta è motivata dal fatto che quando si opera su un computer quantistico reale si utilizzano numerosi qubit, la cui rappresentazione sotto forma di vettore diventerebbe estremamente difficoltosa.

L'algebra di Dirac comprende due tipi di vettori: \textbf{bra} e il suo vettore duale \textbf{ket}.

Un ket rappresenta un vettore colonna e viene utilizzato solitamente per descrivere lo stato di un sistema:
\[
  | a \rangle = \begin{pmatrix} \alpha \\ \beta \end{pmatrix}
\]

Mentre un bra rappresenta la coniugata trasposta del vettore colonna ket:
\[
  \langle a | = \begin{pmatrix} \alpha & \beta \end{pmatrix}
\]

Il prodotto scalare tra i due vettori si indica con \(\langle \alpha | \beta \rangle\) in modo che il prodotto formi un \textbf{braket}.

Definendo due vettori:
\[
  \begin{pmatrix} 1 \\ 0 \end{pmatrix}
  \begin{pmatrix} 0 \\ 1 \end{pmatrix}
\]
e associandoli rispettivamente agli stati \( | 0 \rangle\) e \( | 1 \rangle\), essi formano una base \textit{ortonormale}, cioè una base \textit{ortogonale} di vettori aventi \textit{norma 1}, nota come \textbf{base computazionale standard}.

Possiamo inoltre dare una definizione degli stati attraverso la forma matriciale (vettori colonne), ottenendo la seguente rappresentazione:
\[
  | 0 \rangle = \begin{bmatrix} 1 \\ 0 \end{bmatrix}
  | 1 \rangle = \begin{bmatrix} 0 \\ 1 \end{bmatrix}
\]

I due vettori appena introdotti corrispondono esattamente agli stati classici 0 e 1. A questo punto è bene specificare la principale differenza con il bit classico: un qubit, oltre a potersi trovare in uno degli stati fondamentali, potrà trovarsi contemporaneamente anche in un'altra qualsiasi combinazione di entrambi gli stati base.

Se definiamo \( | \psi \rangle \) la seguente combinazione lineare:
\[
  | \psi \rangle = \alpha | 0 \rangle + \beta | 1 \rangle
\]
dove \(\alpha\) e \(\beta\) rappresentano numeri complessi tali che valga:
\[
  \lvert \alpha \rvert ^2 + \lvert \beta \rvert ^2 = 1
\]
allora \( | \psi \rangle \) è un possibile stato del qubit la cui notazione algebrica equivalente sarà:
\[
  | \psi \rangle
  = \alpha \begin{pmatrix} 1 \\ 0 \end{pmatrix}
  + \beta \begin{pmatrix} 0 \\ 1 \end{pmatrix}
  = \begin{pmatrix} \alpha \\ \beta \end{pmatrix}
\]

Il che equivale a dire che \( | \psi \rangle \) si trova in una sovrapposizione di stati. Quando abbiamo a che fare con un bit classico possiamo sempre stabilire con assoluta certezza in quale dei due stati esso si trovi, nel caso di un qubit non possiamo determinare con altrettanta precisione il suo stato quantistico, ossia i valori esatti di \(\alpha\) e \(\beta\).

La meccanica quantistica ci dice che soltanto attraverso l'effettiva misurazione del sistema otterremo un valore discreto del qubit, più precisamente si dice che lo stato collasserà nello stato \( | 0 \rangle \) con probabilità \( \lvert \alpha \rvert ^2 \) o in \( | 1 \rangle \) con probabilità \( \lvert \beta \rvert ^2 \). Proprio per questa ragione, i due valori \(\alpha\) e \(\beta\) prendono il nome di \textbf{ampiezze di probabilità} (amplitudes). Una prima semplice sovrapposizione è definita dallo stato:
\[
  \frac{1}{\sqrt{2}} | 0 \rangle
  + \frac{1}{\sqrt{2}} | 1 \rangle
  = \frac{1}{\sqrt{2}} (| 0 \rangle + | 1 \rangle)
\]
il quale ci tornerà utile in seguito.

Dunque per ora possiamo immaginare che fino al momento della sua effettiva misurazione, un qubit avrà una probabilità del 50\% di trovarsi nello stato \( | 0 \rangle \) e un altro 50\% di trovarsi in \( | 1 \rangle \); come se lanciando una moneta essa continuasse a girare su sé stessa fino al momento in cui la guardiamo e ne osserviamo il valore.

\subsection{Rappresentazione geometrica di un qubit}
Per ottenere una visualizzazione geometrica utile per comprendere meglio i diversi stati in cui un qubit può trovarsi, utilizziamo una sfera di raggio unitario la cosiddetta \textbf{Sfera di Bloch}, introdotta dal fisico Felix Bloch \cite{bloch_sphere}.
Gli stati del qubit verranno collocati in punti precisi della superficie della sfera, associando quindi ad ogni stato un punto. Lo stato \( | 1 \rangle \) verrà collocato nel polo sud, lo stato \( | 0 \rangle \) nel polo nord. I punti che giacciono sull'equatore avranno una probabilità del 50\% di essere nello stato \( | 0 \rangle \) e 50\% di essere nello stato \( | 1 \rangle \) così le altre locazioni indicheranno gli altri stati di sovrapposizioni quantistiche di \( | 0 \rangle \) e \( | 1 \rangle \).

\begin{figure}[h]
  \centering
  \includegraphics[width=0.7\textwidth]{qubit_geom.png}
  \caption{Rappresentazione di una Sfera di Bloch}
  \label{fig:qubit_geom}
\end{figure}

Come possiamo vedere nella figura \ref{fig:qubit_geom}, possiamo stabilire una corrispondenza biunivoca fra la rappresentazione generica dello stato di un qubit:
\[
  | \psi \rangle
  = \alpha | 0 \rangle
  + \beta | 1 \rangle
\]
E la sua rappresentazione sulla sfera unitaria in \( \mathbb{R} ^3 \):
\[
  | \psi \rangle
  = \cos \left( \frac{\theta}{2} \right)
  | 0 \rangle
  + e^{i\phi}
  \sin \left( \frac{\theta}{2} \right)
  | 1 \rangle
\]
Dove \( \theta \) e \( \phi \) sono le coordinate sferiche del punto. Si può quindi scrivere
\[
  | \psi \rangle
  = \alpha | 0 \rangle
  + e^{i\phi} \beta | 1 \rangle
\]
Dato che il vettore di stato ha norma 1:
\[
  \sqrt{
    |\alpha|^2
    + |\beta|^2
  }
  = 1
\]
si usa l'identità trigonometrica:
\[
  \sqrt{
    \sin^2 x
    + \cos^2 x
  }
  = 1
\]
Per descrivere \( \alpha \) e \( \beta \) reali in termini della variabile \( \theta \):
\[
  \alpha = \cos \left( \frac{\theta}{2} \right), 
  \beta = \sin \left( \frac{\theta}{2} \right)
\]
da questo lo stato di ogni qubit si può descrivere usando le due variabili \(\theta\) e \(\phi\):
\[
  | \psi \rangle
  = \cos \left( \frac{\theta}{2} \right) | 0 \rangle
  + e^{i\phi} \sin \left( \frac{\theta}{2} \right) | 1 \rangle
\]

Interpretanto \(\theta\) e \(\phi\) come coordinate sferiche, si può tracciare qualsiasi stato del qubit sulla superficie della sfera di Bloch. In figura \ref{fig:qubit_stato} vengono visualizzati i seguenti vettori di stato del qubit:

\begin{figure}[h]
  \centering
  \includegraphics[width=0.7\textwidth]{qubit_stato.png}
  \caption{Visualizzazione dei qubit}
  \label{fig:qubit_stato}
\end{figure}

\begin{itemize}
  \item \( \begin{bmatrix} 1 \\ 0 \end{bmatrix} \) con \( \theta = 0 \) e \(\phi = 0\) cioè lo stato \( | 0 \rangle \)
  \item \( \begin{bmatrix} 0 \\ 1 \end{bmatrix} \) con \( \theta = 180 \) e \(\phi = 0\) cioè lo stato \( | 1 \rangle \)
  \item \( \begin{bmatrix} \frac{1}{\sqrt{2}} \\ \frac{i}{\sqrt{2}} \end{bmatrix} \) con \( \theta = \frac{\pi}{2} \) e \( \phi = \frac{\pi}{2} \)
  \item \( \begin{bmatrix} \frac{1}{\sqrt{2}} \\ \frac{1}{\sqrt{2}} \end{bmatrix} \) con \( \theta = \frac{\pi}{2} \) e \( \phi = 0 \) chiamato anche stato \( | + \rangle \)
  \item \( \begin{bmatrix} \frac{1}{\sqrt{2}} \\ \frac{-i}{\sqrt{2}} \end{bmatrix} \) con \( \theta = \frac{\pi}{2} \) e \( \phi = \frac{3\pi}{2} \)
  \item \( \begin{bmatrix} \frac{1}{\sqrt{2}} \\ \frac{-1}{\sqrt{2}} \end{bmatrix} \) con \( \theta = \frac{3\pi}{2} \) e \( \phi = 0 \) chiamato anche stato \( | - \rangle \)
\end{itemize}

Dato che in input inizialmente i qubit hanno sempre stato \( | 0 \rangle \), per poter operare sui qubit e ottenere degli stati differenti bisogna ruotare gli assi cardinali con le apposite \textit{porte logiche quantistiche}.

\section{Porte logiche quantistiche}
Esattamente come nel modello di computazione classica utilizziamo delle porte logiche come l'\textit{AND}, \textit{OR} o il \textit{NOT} per effettuare delle operazioni tra bit, nel modello quantistico avremo delle porte che si occuperanno di manipolare i qubit per ottenere un risultato. In particolare ogni gate quantistico deve rispettare due criteri fondamentali:

\begin{itemize}
  \item \textbf{Reversibilità}: Un qubit a cui è stato applicato un cambiamento dello stato tramite l'utilizzo di una porta deve poter ritornare nello stato iniziale tramite l'applicazione della stessa porta all'output della prima.
  \item \textbf{Conservazione del vincolo di normalizzazione}: In questo modello, le porte logiche sono rappresentate da matrici unitarie. Una matrice quadrata \( U \) viene definita \textbf{unitaria} se vale \( UU^* = I \), dove \( U^* \) è la matrice \textbf{trasposta} e \( I \) è la \textbf{matrice identità}.
\end{itemize}

Proprio come nel modello classico, abbiamo sia porte logiche che agiscono su un singolo qubit, che porte che agiscono su più qubit.

\subsection{Porte Logiche a singolo qubit}
Contrariamente a quanto accade per le porte classiche, in ambito quantistico le porte a singolo bit non si limitano al \textbf{NOT}. Infatti abbiamo in totale quattro porte: \textbf{porta X}, \textbf{porta Y}, \textbf{porta Z} e \textbf{porta di Hadamard}. 

Le porte X, Y e Z prendono il nome di \textit{Porte di Pauli} e corrispondono a delle rotazioni rispettivamente sull'asse x, y e z della sfera di Bloch.

\subsubsection{Porta X}
Analoga alla porta NOT classica, la porta X svolge la stessa operazione del NOT classico invertendo lo stato del qubit nel caso sia uno degli stati base. La differenza con la porta classica sta nel fatto che il NOT nel modello quantistico si dovrà occupare anche di gestire degli stati sovrapposti che sono caratterizzati dai coefficienti \( \alpha \) e \( \beta \) del qubit.
Immaginando di rappresentare in forma vettoriale il qubit, e definendo la matrice corrispondente al NOT quantistico come:
\[
  X = 
  \begin{bmatrix}
    0 & 1 \\
    1 & 0
  \end{bmatrix}
\]
è facilmente verificabile che applicando tale porta a un qubit nella forma \( \alpha | 0 \rangle + \beta | 1 \rangle \) otterremo, seguendo la notazione vettoriale:
\[
  X
  \begin{bmatrix}
    \alpha \\ \beta
  \end{bmatrix}
  =
  \begin{bmatrix}
    \beta \\ \alpha
  \end{bmatrix}
\]

\begin{figure}[h]
  \centering
  \includegraphics[width=0.7\textwidth]{gate_x.png}
  \caption{Visualizzazione dell'applicazione della Porta X}
  \label{fig:gate_x}
\end{figure}

\subsubsection{Porta Y}
La porta Y è rappresentata dalla seguente matrice:
\[
  Y
  =
  \begin{bmatrix}
    0 & -i \\
    -i & 0
  \end{bmatrix}
\]
che mappa la componente \( | 0 \rangle \) in \( i | 1 \rangle \) e la componente \( | 1 \rangle \) in \( -i | 0 \rangle \).

\begin{figure}[h]
  \centering
  \includegraphics[width=0.7\textwidth]{gate_y.png}
  \caption{Visualizzazione dell'applicazione della Porta Y}
  \label{fig:gate_y}
\end{figure}

\subsubsection{Porta Z}
La porta Z è rappresentata dalla seguente matrice:
\[
  Z
  =
  \begin{bmatrix}
    1 & 0 \\
    0 & -1
  \end{bmatrix}
\]
che cambia il segno esclusivamente alla componente nello stato \( | 1 \rangle \).

\begin{figure}[h]
  \centering
  \includegraphics[width=0.7\textwidth]{gate_z.png}
  \caption{Visualizzazione dell'applicazione della Porta Z}
  \label{fig:gate_z}
\end{figure}

\subsubsection{Porta di Hadamard}
La porta di Hadamard è rappresentata dalla seguente matrice:
\[
  H
  =
  \frac{1}{\sqrt{2}}
  \begin{bmatrix}
    1 & 1 \\
    1 & -1
  \end{bmatrix}
\]
che si occupa di trasformare uno stato base in una sovrapposizione di tale stato che si trovi con il 50\% di probabilità in uno dei due stati fondamentali.

\begin{figure}[h]
  \centering
  \includegraphics[width=0.7\textwidth]{gate_h.png}
  \caption{Visualizzazione dell'applicazione della Porta di Hadamard}
  \label{fig:gate_h}
\end{figure}

\subsection{Porte logiche a qubit multipli}
Proprio come nel modello di computazione classico, anche in questo modello siamo interessati ad avere un insieme di gate capaci di realizzare tutte le operazioni del modello classico. Nel caso del modello quantistico, per ottenere tale risultato, si affiancano le porte a singolo qubit con un operatore chiamato \textbf{CNOT} o \textbf{NOT Controllato}.

Il CNOT, che corrisponde allo XOR del modello classico, è dotato di due qubit in ingresso, rispettivamente definiti \textit{controllo} e \textit{bersaglio} (o \textit{target}). Dunque nel caso il qubit controllo si trovi nello stato zero allora il target viene lasciato inalterato, al contrario, se il qubit controllo è nello stato uno, allora il target viene invertito. Tale trasformazione può essere scritta come:
\[
  | A, B \rangle \mapsto | A, B \oplus A \rangle
\]
Il gate è rappresentato dalla seguente matrice:
\[
  CNOT = 
  \begin{bmatrix}
    1 & 0 & 0 & 0 \\
    0 & 1 & 0 & 0 \\
    0 & 0 & 0 & 1 \\
    0 & 0 & 1 & 0
  \end{bmatrix}
\]
Dove effettivamente possiamo notare come gli ultimi due stati vengano rispettivamente invertiti e prendendo in esempio un sistema composto da due qubit, il CNOT eseguirà operazioni mostrate in tabella \ref{tab:cnot}:

\begin{table}[htbp]
  \centering
  \begin{tabular}{|cc|cc|}
  \hline
  \multicolumn{2}{|c|}{Input} & \multicolumn{2}{c|}{Output} \\ \hline
  \multicolumn{1}{|c|}{Controllo} & Target & \multicolumn{1}{c|}{Controllo} & Target \\ \hline
  \multicolumn{1}{|c|}{\(|0\rangle\)} & \(|0\rangle\) & \multicolumn{1}{c|}{\(|0\rangle\)} & \(|0\rangle\) \\ \hline
  \multicolumn{1}{|c|}{\(|0\rangle\)} & \(|1\rangle\) & \multicolumn{1}{c|}{\(|0\rangle\)} & \(|1\rangle\) \\ \hline
  \multicolumn{1}{|c|}{\(|1\rangle\)} & \(|0\rangle\) & \multicolumn{1}{c|}{\(|1\rangle\)} & \(|1\rangle\) \\ \hline
  \multicolumn{1}{|c|}{\(|1\rangle\)} & \(|1\rangle\) & \multicolumn{1}{c|}{\(|1\rangle\)} & \(|0\rangle\) \\ \hline
  \end{tabular}
  \caption{Insieme delle possibili operazioni del gate CNOT}
  \label{tab:cnot}
\end{table}

Una delle proprietà fondamentali delle porte quantistiche, in particolare del CNOT e di tutte le porte viste a singolo qubit, è quella di essere invertibili, infatti a differenza delle porte classiche XOR e NAND generalmente irreversibili, permettono di ottenere l'input avendo a disposizione il valore di output. Combinando opportunamente CNOT e porte a singolo qubit, otteniamo l'insieme dei gate necessari per definire un insieme universale, capace dunque di inglobare le operazioni sufficienti alla rappresentazione di tutte le porte logiche quantistiche e quindi l'universalità delle operazioni quantistiche.

\section{Misurazione di un sistema di qubit}
Fin'ora abbiamo parlato di come vengono effettuate le operazioni sui qubit, tralasciando il modo in cui alla fine della computazione le informazioni sono raccolte. Immaginiamo che una particella sia dotata di un numero finito possibile di stati base e che tale particella li possegga tutti contemporaneamente fin quando non avviene l'evento della misurazione che farà ottenere uno degli stati base con probabilità uguale al quadrato del coefficiente associato a tale stato.

Nel nostro caso dato un qubit \(| \phi \rangle \) generico, il risultato di questa misurazione ci restituisce 0 con probabilità \( |\alpha|^2 \) e 1 con probabilità \( |\beta|^2 \).

Il problema in questo caso è che la misurazione disturba il qubit, lasciandolo nello stato \( | 0 \rangle \)  se il risultato della misurazione è 0, e nello stato \( | 1 \rangle \)  se il risultato della misurazione è 1.

In un circuito quantistico, a differenza della controparte classica, dopo la misurazione di un qubit esso viene scartato in quanto il suo stato essendo collassato, non è più valido.

Altra differenza con la controparte classica è la predicibilità, ovvero che se l'esperimento effettuato venisse ripetuto rispettando le condizioni, ci aspettiamo esattamente lo stesso risultato cosa che in ambito quantistico risulta incerta perché coadiuvata dal coefficiente associato allo stato.

\section{Registri quantistici}
Fin'ora abbiamo visto come rappresentare un solo qubit, per rappresentare un sistema a più qubit si utilizza un \textbf{registro quantistico}, che di fatto indica in che modo i qubit sono collegati tra loro. Per rappresentare questi registri si usa il \textbf{prodotto tensore} \( \otimes \), un operatore che combina spazi vettoriali di una certa dimensione per generarne dei più grandi, infatti: \( \otimes: \mathbb{C}^k \times \mathbb{C}^m \rightarrow \mathbb{C}^{km} \) quindi lo spazio totale di un registro quantistico sarà \( \mathbb{C}^{2 \cdot...\cdot 2} = \mathbb{C}^{2^n} \).

Formalmente si definisce un registro quantistico, secondo il quarto postulato della meccanica quantistica\footnote{Lo spazio degli stati di un sistema fisico composto è il prodotto tensore degli spazi degli stati dei sistemi fisici componenti. Se il sistema è composto da n sottosistemi e il componente i-esimo si trova nello stato \( | \phi _i \rangle \) allora lo stato del sistema totale è \( | \phi _1 \rangle \otimes | \phi _2 \rangle \otimes ... | \phi _n \rangle \)}, come:
\[
  | i _1 \rangle \otimes | i _2 \rangle \otimes ... | i _n \rangle
\]
dove \(i = 0,1\) e \(n\) è il numero di qubit e per convenienze possiamo rappresentare questo vettore semplicemente come \( | i _1 i _2 ... i _n \rangle \). Consideriamo un semplice sistema a due qubit, dove il primo è \( |\psi \rangle  = \alpha _0 |0 \rangle  + \beta _0 |1 \rangle \)  mentre il secondo \( |\theta \rangle  = \alpha _1 | 0 \rangle  + \beta _1 | 1 \rangle \). Lo stato totale sarà una sovrapposizione dalla forma:
\[
  | \psi \rangle \otimes | \phi \rangle 
  = \alpha _{01} | 00 \rangle 
  + \alpha _0 \beta _1 | 01 \rangle 
  + \alpha _1 \beta _0 | 10 \rangle 
  +  \beta _{01} | 11 \rangle  
\]

Analogamente al singolo qubit dove il risultato della misurazione ci restituisce 0 con probabilità \( |\alpha|^2 \) e 1 con probabilità \( |\beta|^2 \). In un sistema di \(n\) qubit possiamo anche misurare solo un sottoinsieme degli \(n\) qubit. Ad esempio lo stato risulterà in \(| 00 \rangle \) con probabilità \( | \alpha _{01}|^2 \), in \(| 01 \rangle \) con probabilità \( | \alpha _0 \beta _1|^2 \) e così via. Inoltre se volessimo sapere la probabilità di ottenere 0 al primo bit basta sommare le probabilità di \(| 00 \rangle \) e \(| 01 \rangle \) cioè \( |\alpha _{01}|^2 + |\alpha _0 \beta _1|^2 \).

\section{Entanglement}
Dopo aver visto i registri quantistici una ulteriore proprietà legata ai possibili stati in cui può trovarsi il sistema è l'\textit{entanglement}, proprietà che non possiamo ritrovare in nessun oggetto della fisica classica. Questi stati chiamati entangled rappresentano quelle possibili configurazioni di n qubit componenti che non hanno un proprio stato ben definito ma solamente la loro combinazione ne rappresenta uno concreto. Più semplicemente uno stato entangled non può essere descritto come prodotto tensore degli stati dei singoli componenti.
Gli stati entangled si comportano come se fossero strettamente connessi l'uno all'altro indipendentemente dalla distanza fisica che li separa, in modo che una misurazione o un'operazione di uno dei due stati di una coppia entangled fornisce simultaneamente informazioni sulla coppia.
Un esempio per spiegare questa proprietà è dato dallo stato \( | 00 \rangle + | 11 \rangle \) che non può essere fattorizzato nel prodotto tensore di due qubit indipendenti, in quanto non esistono dei coefficienti \( \alpha _1 \alpha _2 \beta _1 \beta _2 \) tali per cui valga:
\[
  | 00 \rangle + | 11 \rangle
  = ( \alpha _1 | 0 \rangle + \beta _1 | 1 \rangle) \otimes ( \alpha _2 | 0 \rangle + \beta _2 | 1 \rangle)  
\]

\begin{figure}[h]
  \centering
  \includegraphics[width=1\textwidth]{entanglement.png}
  \caption{Visualizzazione degli effetti dell'entanglement}
  \label{fig:entanglement}
\end{figure}

L'entanglement è alla base della risoluzione di alcuni di quei problemi informatici non riproducibili tramite informatica classica, grazie alla sua intrinseca proprietà, non esistente nella fisica classica, che da possibilità di ottenere un aumento esponenziale nella capacità di calcolo.

\section{Realizzazione di un computer quantistico}
Definiamo un computer quantistico come un calcolatore che segue il modello di computazione quantistico, sfruttando i dettami della fisica quantistica per eseguire dei calcoli che in alcuni casi risultano essere impossibili da realizzare in un calcolatore classico.

\subsection{Classi di complessità}
Prima di proseguire con l'introduzione delle componenti di un computer quantistico e bene tenere a mente le classi di complessità che non sono altro un insieme di problemi di una determinata complessità.
I problemi vengono eseguiti su una macchina di Turing per individuarne la particolare classe di complessità in cui rientrano. Le due classi più importanti sono \textbf{P} e \textbf{NP}.

\begin{figure}[h]
  \centering
  \includegraphics[width=0.7\textwidth]{complexity.png}
  \caption{Le classi di complessità per \(P = NP\) e \(P \neq NP\)}
  \label{fig:complexity}
\end{figure}

\begin{itemize}
  \item La classe \textbf{P} è l'insieme dei problemi di decisione che possono essere risolti da una macchina di Turing deterministica in tempo polinomiale.
  \item La classe \textbf{NP} è l'insieme dei problemi di decisione che possono essere risolti da una macchina di Turing non deterministica in tempo polinomiale. Inoltre nella classe NP è composta anche dalle classi NP-Complete e NP-Hard.
  \begin{itemize}
    \item La classe \textbf{NP-Complete} è l'insieme dei problemi più difficili nella classe NP nel senso che, se si trovasse un algoritmo in grado di risolvere "velocemente" (in tempo polinomiale) un qualsiasi problema NP-completo, allora si potrebbe usarlo per risolvere "velocemente" ogni problema in NP.
    \item In teoria della complessità, i problemi NP-difficili o NP-ardui sono una classe di problemi che può essere definita informalmente come la classe dei problemi almeno difficili come i più difficili problemi delle classi di complessità P e NP.
  \end{itemize}
\end{itemize}

Gli informatici \textit{Bernstein} e \textit{Vazirani} nel 1997 definirono una nuova classe di complessità chiamata \textbf{BQP} \cite{bernstein1997quantum} (\textit{Bounded-error Quantum Polynomial time}) che è la classe di complessità dei problemi decisionali che possono essere risolti con un errore bilaterale su una macchina di Turing quantistica in tempo polinomiale. In breve, tutti i problemi decisionali che i computer quantistici possono risolvere in maniera veloce. Inoltre, è stato dimostrato che \( P \in BPQ \) e quindi è semplice dedurre che i computer quantistici possono riolvere tutti i problemi che i computer classici possono risolvere.

\begin{figure}[htbp]
  \centering
  \includegraphics[width=1\textwidth]{bqp.png}
  \caption{La classe di complessità BQP rispetto a quelle classiche}
  \label{fig:bqp}
\end{figure}

\subsection{Macchina di Turing Quantistica}
La \textbf{Macchina di Turing Quantistica} (\textbf{QTM}) è stata descritta per la prima volta da Deutsch \cite{deutsch1985quantum}. L'idea di base è abbastanza semplice, un QTM è più o meno una Macchina di Turing probabilistica (PTM) con ampiezze di transizione complesse anzichè probabilità reali. A sua volta una Macchina di Turing Probabilistica (PTM) è identica a una normale Macchina di Turing tranne per il fatto che ad ogni configurazione della macchina (\(q_{i}S_{j}\)) c'è un insieme finito di regole di transizione (ognuna con una probabilità associata) che si applicano e che una scelta casuale determina quale regola applicare. Fissiamo una soglia di probabilità maggiore delle quote pari (diciamo, 75\%) e diciamo che una PTM specifica calcola \(f(x)\) sull'input \(x\) se e solo se si ferma con \(f(x)\) come output con probabilità maggiore del 75\%.

\subsection{Condizioni per la realizzazione}
Per la realizzazione di un computer quantistico nel 2000 sono stati stilati dal fisico teorico Di Vincenzo i \textbf{criteri di DiVincenzo} \cite{DiVincenzo_2000} che consistono in sette condizioni necessarie per costruire un computer seguentdo il modello quantistico, le prime cinque sono necessarie per il calcolo quantistico e sono:

\begin{enumerate}
  \item Il sistema deve essere \textit{scalabile}, con qubit ben caratterizzati;
  \item Deve essere possibile preparare uno \textit{stato iniziale generico}, ad esempio \(| 0000 \rangle \). Diversamente, sarà impossibile introdurre dati nel computer;
  \item I \textit{tempi di de-coerenza} devono essere abbastanza lunghi, per poter realizzare un numero sufficiente di operazioni sfruttando la correlazione quantistica;
  \item Occorre \textit{un insieme universale di porte quantistiche}, ovverosia si deve poter costruire una varietà sufficiente di porte quantistiche per permettere qualsiasi operazione logica;
  \item Si deve disporre di un modo per misurare lo stato dei qubit, senza il quale sarebbe impossibile estrarre l'informazione processata dal computer;
\end{enumerate}

Le restanti due servono per la comunicazione quantistica e sono:

\begin{enumerate}
  \setcounter{enumi}{5}
  \item Deve esserci un sistema per \textit{convertire i qubit} immagazzinati in qubit messaggeri, ovvero deve esistere un sistema per trasmettere informazioni;
  \item La capacità di \textit{trasmettere fedelmente} qubit tra le varie locazioni specificate, per la medesima ragione del punto precedente.
\end{enumerate}

Negli ultimi anni si stanno sperimentando vari modi per la realizzazione dei computer quantistici come:
\begin{itemize}
  \item Superconduttori
  \item Ioni intrappolati di un atomo o di una molecola
  \item Risonanza magnetica nucleare
  \item Quantum annealing o ricottura quantistica
  \item Silicium quantum dot
\end{itemize}

\begin{figure}[htbp]
  \centering
  \includegraphics[width=1\textwidth]{qubit_how.png}
  \caption{Approcci per la realizzazione di qubit}
  \label{fig:qubit_how}
\end{figure}

Tra tutti i più utilizzati dai produttori come IBM, Google, Rigetti sono:

\begin{description}
  \item[Ioni intrappolati di un atomo] In questa tipologia di approccio, vengono costruite delle cosiddette \textit{ion trap} o trappole di ioni. Il loro scopo è trattenere all'interno degli ioni, come ad esempio un atomo di calcio che tramite l'utilizzo di un raggio laser è stato privato di uno dei due elettroni più esterni. Un chip costruito con questo approccio dei qubit è molto simile ai chip di cui sono composte le CPU classiche: si tratta infatti di un chip composto di oro su cui sono presenti gli ioni di calcio e al di sopra di essi, circa ad una distanza pari al diametro di un capelli, è presente un sottile strato di oro che alternando appositamente il suo campo magnetico, riesce a tenere gli ioni nella loro posizione ed evitare che fuoriescano (da qui si capisce il termine ion trap). \\
  Come è possibile trattare questi ioni come qubit? Innanzitutto, gli ioni naturalmente seguono i principi della meccanica quantistica, ed è possibile ottenere i due stati base di un qubit tramite l'utilizzo dello spin, presente in ogni atomo, che rappresenta una intrinseca forma di momento angolare di una particella elementare. Possiamo immaginare lo spin dell'elettrone del nostro atomo di calcio come un magnete: il nord può puntare verso l'alto, ottenendo un qubit in uno stato \(|1\rangle\) che in questo caso corrisponde a \(|\uparrow\rangle\), oppure verso il basso, ottenendo uno stato \(|0\rangle\) corrispondente a \(|\downarrow\rangle\). Per passare fra lo stato \(|\uparrow\rangle\) e \(|\downarrow\rangle\), basta utilizzare delle microonde che hanno l'effetto di ruotare lo spin dell'elettrone. È possibile quindi ruotare e fermarci in un qualsiasi stato compreso fra i due spin, ottenendo quella che abbiamo in precedenza chiamato superposizione.
  \item[Superconduttori] L'approccio che utilizza i superconduttori per costruire i qubit viene utilizzato ad esempio nei computer quantistici di Google o IBM. Proprio con questo metodo di costruzione, nel 2016 Google ha annunciato di aver raggiunto la Quantum Supremacy \cite{quantum_supremacy}, cioè è stato risolto un problema che nessun calcolatore classico potrebbe risolvere in un ragionevole lasso di tempo, utilizzando un computer quantistico a 56 qubit, prodotti con questo approccio. Un superconduttore è un particolare materiale che raffreddato ad una temperatura molto vicina allo zero assoluto (0K oppure -273.15C) annulla la sua resistività elettrica completamente e grazie a queste particolari caratteristiche risulta adatto per essere utilizzato come qubit.
\end{description}

\chapter{Blockchain}
Alla base della più moderna forma di commercio, incentrata sulle criptovalute, troviamo una delle forme di commercio più antica mai messa agli atti. Infatti, il viaggio all'interno della Blockchain e le criptovalute ha inizio nel 1400 d.C. in una piccola isola della Micronesia, l'isola di Yap.

\section{La storia della Blockchain}

L'evoluzione della blockchain può essere riassunta nei seguenti passaggi principali mostrati nella tabella temporale \ref{tab:blockchain_evolution}.

Nel 1982, il crittografo David Chaum ha proposto per la prima volta un protocollo simile alla blockchain nella sua tesi del 1982 \textit{"Computer e sistemi creati, mantenuti e resi attendibili da gruppi di individui reciprocamente sospettosi"} \cite{computer_systems_chaum}, da qui in poi li definiamo \textbf{Sistemi di Chaum}. Siamo così difronte alla prima idea di tecnologia blockchain.

\begin{table}[htbp]
  \centering
  \scalebox{1.2}{
    \begin{tabular}{r | @{\foo} l}
      1982 & Sistemi di Chaum \newline \\
      1991 & Timestamp \newline \\
      1992 & Alberi di Merkle \newline \\
      2005 & Bitgold \newline \\
      2008 & Bitcoin \\
    \end{tabular}
  }
  \caption{Evoluzione della Blockchain}
  \label{tab:blockchain_evolution}
\end{table}

\subsection{Introduzione ai Sistemi di Chaum}
Probabilmente, molti degli elementi delle blockchain odierne sono contenuti nel sistema di caveau di David Chaum del 1979, descritto nella sua tesi di laurea del 1982 a Berkeley. Chaum descrive la progettazione di un sistema informatico distribuito che può essere creato, mantenuto e reso attendibile da gruppi di individui reciprocamente sospettosi.

Si tratta di un sistema contenente record in grado di manetere la sicurezza e la privacy dei singoli individui tramite sicurezza fisica. Gli elementi costitutivi di questo sistema includono "caveau" fisici (sicuri), primitive crittografiche (crittografia simmetrica e asimmetrica, funzioni hash crittografiche e firme digitali), e una nuova primitia introdotta da Chaum.

\subsection{Timestamp}
Un ulteriore lavoro su una catena di blocchi protetta da crittografia è stato descritto nel 1991 da Stuart Haber e W. Scott Stornetta \cite{haber1990time}. Essi volevano implementare un sistema in cui i timestamp dei documenti non potessero essere manomessi, oggi considerata la prima applicazione della blockchain.

L'utilizzo del timestamp richiede il superamento di due problematiche:

\begin{itemize}
  \item I dati DEVONO essere contrassegnati con l'ora esatta
  \item Il calendario DEVE essere immutabile
\end{itemize}

I due, idearono una soluzione a queste problematiche, definita "naive", la quale consisteva nell'utilizzo di una \textit{cassetta di sicurezza digitale}. Ogni volta che un cliente ha un documento da marcare temporalmente, lo trasmette a un servizio di marcatura temporale (TSS). Il servizio registra la data e l'ora di ricezione del documento e ne conserva una copia. Se l'integrità del documento del cliente viene messa in discussione, viene confrontata con la copia conservata dal TSS. Se le due copie sono identiche, è la prova che il documento non è stato manomesso dopo la data riportata nei registri del TSS.

Questa procedura soddisfa di fatto il requisito centrale per la marcatura temporale di un documento digitale. Tuttavia, questo approccio solleva diverse preoccupazioni:

\begin{description}
  \item[Privacy] Questo metodo compromette la privacy del documento in due modi: una terza parte potrebbe origliare mentre il documento viene trasmesso e, dopo la trasmissione, il documento è a disposizione del TSS stesso. Il cliente deve quindi preoccuparsi non solo della sicurezza dei documenti che tiene sotto il suo diretto controllo, ma anche della sicurezza dei suoi documenti presso il TSS.
  \item[Larghezza di banda e archiviazione] Sia il tempo necessario per inviare un documento per la marcatura temporale che la quantità di memoria richiesta al TSS dipendono dalla lunghezza del documento da marcare. Pertanto, il tempo e la spesa necessari per la marcatura temporale di un documento di grandi dimensioni potrebbero essere proibitivi. 
  \item[Incompetenza] La copia del documento inviata al TSS potrebbe essere danneggiata durante la trasmissione al TSS, potrebbe essere marcata in modo errato quando arriva al TSS, oppure potrebbe essere danneggiata o persa del tutto in qualsiasi momento mentre è conservata presso il TSS. Ognuno di questi eventi invaliderebbe la richiesta di marcatura temporale del cliente.
  \item[Fiducia] Il problema fondamentale rimane: nulla in questo schema impedisce al TSS di accordarsi con un cliente per affermare di aver apposto la data e l'ora su un documento diverso da quello reale.
\end{description}

Per risolvere queste criticità, Haber e Stornetta, formularono una soluzione: proposero di sottoporre il documento ad un algoritmo di hashing crittografico, ottenendo così un ID univoco ed immutabile del documento.
Semplicemente, anzichè trasmettere al TSS il documento x, viene trasmesso il suo valore \(hash(x) = y\). Per quanto riguarda l'autenticazione, il timestamp di y sarà valido quanto il timestamp di x. Inoltre, questa soluzione riduce drasticamente il problme della larghezza di banda e dell'archiviazione e in più risolve anche il problema della privacy in quanto non viene trasmesso il documento in toto. A seconda degli obiettivi di progettazione, potrebbe essere una singola funzione di hash comune o una per ogni singola utenza.

A ciò si abbinava la firma digitale, utilizzata per identificare in modo univoco il firmatario. Controllando la firma, al client viene garantito che il TSS abbia elaborato la richiesta, che l'hash sia stato ricevuto correttamente e che l'ora inclusa sia corretta. Questo risolve il problema dell'incompetenza da parte del TSS.

Nella figura \ref{fig:blockchain_struttura} è riportata una sequenza d'esempio in cui abbiamo una catena di blocchi connessi da un valore hash.

\begin{figure}[h]
  \centering
  \includegraphics[width=0.8\textwidth]{blockchain_struttura.png}
  \caption{Sequnza di blocchi}
  \label{fig:blockchain_struttura}
\end{figure}

In questa sequenza di blocchi, ogni documento digitale è modificato dai client in diversi istanti di tempo e la catena mantiene un elenco di valori di timestamp relativi agli eventi accaduti sequenzialmente. I valori di timestamp non sono modificabili e in caso di controversie ogni modifica apportata al documento può essere consultata.

\subsection{Alberi di Merkle}
Dave Bayer, contribuì ad integrare la struttura per la marcatura temporale di Haber e Stornetta, con la realizzazione dei Merkle Tree (Alberi di Merkle) \cite{bayer1993improving}, offrendo l'opportunità di raccogliere più documenti in un singolo blocco (Figura \ref{fig:merkle_tree}). Tali alberi ricevono il nome da Ralph Merkle e in essi i nodi foglia sono contrassegnati da un blocco dati, mentre i nodi non-foglia dall'hash crittografico delle etichette dei loro nodi figlio. Detti anche Alberi di hash, mostrano una versione più generica di liste e catene hash e consentono una verifica sicura ed efficace del contenuto di grandi strutture dati.

\begin{figure}[h]
  \centering
  \includegraphics[width=0.9\textwidth]{merkle_tree.png}
  \caption{Esempio di albero di Merkle}
  \label{fig:merkle_tree}
\end{figure}

Nella figura \ref{fig:merkle_tree} possiamo vedere come i valori hash dei blocchi sono definiti "foglie", mentre i valori hash dei loro figli sono detti "nodi". Gli alberi di Merkle vengono utilizzati per rilevare incongruenze tra le repliche e per ridurre al minimo la quantità di dati.

\subsection{Bit gold}
Nel 2005, si ha avuto il primo tentativo di moneta decentralizzata grazie all'informatico Nick Szabo, il quale ha proposto una nuova valuta basata sulla blockchain: \textbf{Bit gold} \cite{szabo_2005}. Moneta che però non ha riscosso molto successo, ma nonostante ciò il 2005 rappresenta un anno cruciale nel contesto blockchain.

La proposta dell'informatico si basa sul calcolo di una stringa di bit a partire da una stringa di bit di sfida, utilizzando funzioni chiamate in vario modo "client puzzle function", "proof of work function" o "secure benchmark function". La stringa di bit risultante è la proof of work.

Ecco le fasi principali del sistema bit gold che Szabo ha definito:

\begin{enumerate}
  \item Viene creata una stringa pubblica di bit, la "stringa di sfida" (vedi passo 5).
  \item Alice sul suo computer genera la stringa di proof of work dai bit di sfida utilizzando una funzione di benchmark.
  \item La proof of work viene registrata in modo sicuro con un timestamp. Questo dovrebbe funzionare in modo distribuito, con diversi servizi di timestamp in modo che non sia necessario affidarsi a un particolare servizio di timestamp.
  \item Alice aggiunge la stringa di sfida e la stringa di proof of work con timestamp a un registro di proprietà distribuito per il bit gold. Anche in questo caso, non si fa affidamento su un singolo server per il corretto funzionamento del registro.
  \item L'ultima stringa creata di bit gold fornisce i bit di sfida per la stringa creata successivamente.
  \item Per verificare che Alice sia la proprietaria di una particolare stringa di bit gold, Bob controlla la catena di titoli non falsificabile nel registro dei titoli di bit gold.
  \item Per verificare il valore di una stringa di bit gold, Bob controlla e verifica i bit di sfida, la stringa di proof of work e il timestamp.
\end{enumerate}

Si noti che il controllo di Alice sul suo bit gold non dipende dal suo solo possesso dei bit, ma piuttosto dalla sua posizione di leader nella catena di titoli non falsificabile (catena di firme digitali) nel registro dei titoli.

Tutto questo può essere automatizzato da un software. I limiti principali alla sicurezza dello schema sono la capacità di distribuire la fiducia nelle fasi (3) e (4) e il problema dell'architettura della macchina, che verrà discusso di seguito.

Hal Finney ha implementato una variante di bit gold chiamata \textbf{RPOW (Reusable Proofs of Work)}. Si basa sulla pubblicazione del codice informatico della "zecca", che viene eseguito su un computer remoto a prova di manomissione. L'acquirente di bit gold può quindi utilizzare l'attestazione remota, che Finney chiama tecnica del server trasparente, per verificare che un determinato numero di cicli sia stato effettivamente eseguito.

Il problema principale di tutti questi schemi è che gli schemi di proof of work dipendono dall'architettura del computer, non solo da una matematica astratta basata su un "ciclo di calcolo" astratto. (Pertanto, potrebbe essere possibile essere un produttore a bassissimo costo (di diversi ordini di grandezza) e inondare il mercato di bit gold. Tuttavia, dal momento che il bit gold è marcato a tempo, il tempo creato e la difficoltà matematica del lavoro possono essere dimostrati automaticamente. Da ciò si può solitamente dedurre il costo di produzione in quel periodo.

A differenza degli atomi d'oro fungibili, ma come nel caso degli oggetti da collezione, una grande disponibilità in un determinato periodo di tempo farà scendere il valore di questi particolari oggetti. Da questo punto di vista, il "bit gold" si comporta più come gli oggetti da collezione che come l'oro. Tuttavia, la corrispondenza tra questo mercato ex post e l'asta che determina il valore iniziale potrebbe creare un profitto molto consistente per il "minatore di bit gold" che inventa e distribuisce un'architettura informatica ottimizzata.

Pertanto, il bit gold non sarà fungibile in base a una semplice funzione, ad esempio, della lunghezza della stringa. Per creare unità fungibili, i commercianti dovranno invece combinare unità di valore diverso.

\subsection{Bitcoin}
Bitcoin nasce ufficialmente agli inizi del 2009 con la creazione del "blocco genesi", ma se ne inizia a parlare nel 2008 a seguito della pubblicazione di un paper scientifico intitolato \textit{"Bitcoin: A Peer-to-Peer Electronic Cash System"} \cite{bitcoin-white-paper}.

\begin{figure}[h]
  \centering
  \includegraphics[width=0.9\textwidth]{bitcoin_genesis_block.jpeg}
  \caption{Messaggio di Satoshi Nakamoto incorporato nella coinbase del primo blocco}
  \label{fig:bitcoin_genesis_block}
\end{figure}

Il \textbf{libro bianco o "whitepaper" di Bitcoin} fu pubblicato in un articolo scientifico tramite Cryptography Mailing List nel mese di \textbf{ottobre del 2018}. Essendo pubblicato in modo anonimo sotto lo pseudonimo Satoshi Nakamoto genera ancora più mistero e confusione. Così tanto che ancora oggi si cerca un vero nome dietro quel soprannome.

\textit{Prima di iniziare è bene fare una precisazione: bitcoin con la b minuscola è la moneta digitale, Bitcoin con la b maiuscola è il protocollo che la governa.}

L'obiettivo di Satoshi era quello di creare un sistema di pagamento tramite una versione puramente peer-to-peer di denaro elettronico che permetterebbe di effettuare pagamenti online da un'entità ad un'altra senza passare tramite un'istituzione finanziaria centrale. I nodi peer-to-peer, che costituiscono la rete, non formano gerarchie client-server ma agiscono al contempo sia da client che da server.

Le firme digitali offrono una soluzione parziale al problema, ma i benefici principali sono persi se una terza persona di fiducia è ancora richiesta per prevenire la doppia spesa. Ovvero, quando un utente fa una transazione ci deve essere la garanzia che i soldi appena spesi non possano essere utilizzati una seconda volta per compierne un'altra, problema illustrato in Figura \ref{fig:double_spending}.

La moneta fisica risolve alla radice questo problema non potendo esistere in due luoghi contemporaneamente. In merito ai pagamenti digitali, in un sistema di fiducia centralizzato il problema è gestito da una terza parte che fa controlli su ogni operazione effettuata dagli utenti.

\begin{figure}[htbp]
  \centering
  \includegraphics[width=0.9\textwidth]{double_spending.png}
  \caption{Problema del double spending}
  \label{fig:double_spending}
\end{figure}

Satoshi propone una soluzione al problema della doppia spesa mediante l'utilizzo di una rete peer-to-peer, includendo elementi di crittografia.

La legge di mercato della domanda e offerta determinano il valore economico assunto dal bitcoin. Bitcoin è quotato su siti appositi chiamati Exchange. Tali siti permettono di scambiare Bitcoin con Euro, Dollaro Americano o altre monete emesse dai governi, dette anche \textbf{monete fiat}\footnote{Moneta legale (o moneta a corso legale o, ancora, moneta fiduciaria)}. Il primo Exchange è andato online nel marzo del 2010 e quotava bitcoin a soli 0,003\$. Il 22 maggio 2010 vengono acquistate due pizze in Florida per 10.000,00 bitcoin. Meno di un anno dopo la criptomoneta raggiunge il valore di 1,00\$. Nel 2013 la valutazione subisce alti e bassi arrivando a toccare un massimo di 1200,00\$. Con l'apertura di nuovi exchange e grazie alla speculazioni da parte di un numero sempre maggiore di utenti il prezzo sale fino a 10.000,00\$ a novembre 2017. Oggi (fine luglio 2022) ha un valore di 20.893,00\$ circa. L'andamento del bitcoin è illustrato in Figura \ref{fig:bitcoin_quotation}.

\begin{figure}[htbp]
  \centering
  \includegraphics[width=0.9\textwidth]{bitcoin_quotation.png}
  \caption{Andamento del bitcoin}
  \label{fig:bitcoin_quotation}
\end{figure}

\section{Cos'è la blockchain}
La blockchain è una sottofamiglia di tecnologie in cui il registro è strutturato come una catena di blocchi contenenti le transazioni e la cui validazione è affidata a un meccanismo di consenso, distribuito su tutti i nodi della rete nel caso delle \textit{blockchain permissionless o pubbliche} o su tutti i nodi i nodi che sono autorizzati a partecipare al processo di validazione delle transazioni da includere nel registro nel caso delle \textit{blockchain permissioned o private}.

Per alcuni, la blockchain è la nuova generazione di Internet, o meglio ancora è la Nuova Internet. Si ritiene che possa rappresentare una sorta di Internet delle Transazioni arrivndo a creare e rappresentare la Internet del Valore sulla base di sette caratteristiche \cite{bellini_2021}:
\begin{enumerate}
  \item Decentralizzazione
  \item Trasparenza
  \item Sicurezza
  \item Immutabilità
  \item Consenso
  \item Responsabilità
  \item Programmabilità
\end{enumerate}

\subsection{Architettura della blockchain}

\subsection{Come funziona una Blockchain}

\section{Algoritmi di Consenso}

\subsection{Proof of Work}

\subsection{Proof of Stake}

\section{Blockchain pubbliche e private}

\subsection{Blockchain pubbliche}

\subsection{Blockchain private}

\section{Generazioni di Blockchain}

\subsection{Prima generazione: criptovalute}

\subsection{Seconda generazione: digital assets, smart contract e dApp}

\subsection{Terza generazione: scalabilità, interoperabilità e IoT}

\chapter{Blockchain nell'era quantistica}
La Blockchain è indiscutibilmente una delle tecnologie più recenti e fiorente degli ultimi dieci anni, se non la tecnologia del futuro. Però a minacciare la sicurezza di quest'utlima è l'ormai incombente crescita di un ulteriore tecnologia: il Quantum Computing.

In particolare, algoritmi quantistici come l'\textit{algoritmo di fattorizzazione di Shor} e l'\textit{algoritmo di ricerca di Grover}, che sono alla base dell'odierna Blockchain, possono risolvere alcuni problemi in tempi considerevolmente minori rispetto alle loro controparti tradizionali, dando quindi la possibilità di violare, utilizzando migliaia di qubit, schemi di crittografia a chiave pubblica, come \textit{RSA} ed \textit{Elliptic Curve}, essenziali per la sicurezza della Blockchain. Nel 2015, infatti, il \textit{National Institute of Standards and Tecnology (NIST)} degli Stati Uniti ha dichiarato che la tecnologia quantistica ha il 15\% di probabilità di rompere lo schema crittografico RSA 2048 entro il 2026 e il 50\% di possibilità che ciò avvenga entro il 2031, ma che, entro il 2035, sarà sufficientemente avanzata per romperlo definitivamente \cite{mosca2018cybersecurity}.

Una soluzione a questo problema è la \textit{Post-Quantum Cryptography (PQC)} che ha come scopo quelllo di sviluppare dei sistemi crittografici in grado di resistere ad attacchi provenienti da computer quantistici e classici, e che allo stesso tempo si interfaccino con le attuali reti e protocolli di comunicazione. A partire da questo, il NIST ha iniziato un processo di ricerca, valutazione e standardizzazione di uno o più algoritmi di crittografica QR\footnote{Quantum Resistant, o resistente agli attacchi quantistici.} \cite{quantum_NIST}.

\section{Crittografia post-quantistica}
In crittografia, la \textbf{crittografia post-quantistica} (talvolta definita anche \textbf{quantum-resistant}) si riferisce ad algoritmi crittografici (solitamente algoritmi a chiave pubblica) che si ritiene siano sicuri contro un attacco crittoanalitico da parte di un computer quantistico. Come anticipato, il problema degli algoritmi attualmente in uso è che la loro sicurezza si basa su uno dei tre problemi matematici più difficili: il problema della fattorizzazione dei numeri interi, il problema del logaritmo discreto o il problema del logaritmo discreto a curva ellittica.

\subsection{Algoritmi}
Attualmente la ricerca sulla crittografia post-quantistica si concentra principalmente su sei diversi approcci:
\begin{description}
  \item[Crittografia basata su reti euclidee] Questo approccio comprende sistemi crittografici come l'\textit{apprendimento con errori}, l'\textit{apprendimento ad anello con errori (ring-LWE)}, lo \textit{scambio di chiavi ad anello con errori} e la \textit{firma ad anello con errori}, i vecchi schemi di crittografia \textit{NTRU} o \textit{GGH} e le più recenti firme \textit{NTRU} e \textit{BLISS}.
  \item[Crittografia basata sui polinomi multivariati] Questo approccio include sistemi crittografici come lo schema \textit{RAINBOW} (\textit{Unbalanced Oil and Vinegar}) che si basa sulla difficoltà di risolvere sistemi di equazioni multivariate.
  \item[Crittografia basata su hash] Questo approccio include sistemi crittografici come le \textit{firme Lamport}, lo \textit{schema di firma Merkle}, l'\textit{XMSS}, lo \textit{SPHINCS}, e gli schemi \textit{WOTS}.
  \item[Crittografia basata sui codici di correzione degli errori] Questo approccio include sistemi crittografici che si basano su \textit{codici a correzione di errore}, come gli algoritmi di crittografia \textit{McEliece} e \textit{Niederreiter} e il relativo schema di firma \textit{Courtois, Finiasz e Sendrier}.
  \item[Crittografia isogenica a curva ellittica supersingolare] Questo sistema crittografico si basa sulle proprietà delle \textit{curve ellittiche supersingolari} e dei \textit{grafi isogenici supersingolari} per creare una sostituzione Diffie-Hellman con segretezza in avanti.
  \item[Crittografia basata su chiavi simmetriche] Se si utilizzano chiavi di dimensioni sufficientemente grandi, i sistemi crittografici a chiave simmetrica come \textit{AES} e \textit{SNOW 3G} sono già resistenti agli attacchi di un computer quantistico. Inoltre, i sistemi e i protocolli di gestione delle chiavi che utilizzano la crittografia a chiave simmetrica anziché quella a chiave pubblica, come \textit{Kerberos} e la \textit{Mobile Network Authentication Structure del 3GPP}, sono intrinsecamente sicuri contro gli attacchi di un computer quantistico.
\end{description}

\subsection{Confronto}
Una caratteristica comune a molti algoritmi di crittografia post-quantistica è che richiedono chiavi di dimensioni maggiori rispetto agli algoritmi a chiave pubblica "pre-quantistica" comunemente utilizzati. Spesso è necessario trovare un compromesso tra la dimensione della chiave, l'efficienza computazionale e la dimensione del testo cifrato o della firma. La tabella \ref{tab:confronto} elenca alcuni valori per diversi schemi a un livello di sicurezza post-quantistico di 128 bit.

\begin{table}[]
  \resizebox{\columnwidth}{!}{
    \begin{tabular}{|l|l|l|l|l|}
    \hline
    \textbf{Algorithm}                  & \textbf{Type}  & \textbf{Public Key} & \textbf{Private Key} & \textbf{Signature} \\ \hline
    NTRU Encrypt                        & Lattice        & 766.25 B            & 842.875 B            &                    \\ \hline
    Streamlined NTRU Prime              & Lattice        & 154 B               &                      &                    \\ \hline
    Rainbow                             & Multivariate   & 124 KB              & 95 KB                &                    \\ \hline
    \textbf{SPHINCS}                             & \textbf{Hash Signature} & \textbf{1 KB}                & \textbf{1 KB}                 & \textbf{41 KB}              \\ \hline
    SPHINCS+                            & Hash Signature & 32 B                & 64 B                 & 8 KB               \\ \hline
    BLISS-II                            & Lattice        & 7 KB                & 2 KB                 & 5 KB               \\ \hline
    GLP-Variant GLYPH Signature         & Ring-LWE       & 2 KB                & 0.4 KB               & 1.8 KB             \\ \hline
    NewHope                             & Ring-LWE       & 2 KB                & 2 KB                 &                    \\ \hline
    Goppa-based McEliece                & Code-based     & 1 MB                & 11.5 KB              &                    \\ \hline
    Random Linear Code based encryption & RLCE           & 115 KB              & 3 KB                 &                    \\ \hline
    Quasi-cyclic MDPC-based McEliece    & Code-based     & 1,232 B             & 2,464 B              &                    \\ \hline
    SIDH                                & Isogeny        & 564 B               & 48 B                 &                    \\ \hline
    SIDH (compressed keys)              & Isogeny        & 330 B               & 48 B                 &                    \\ \hline
    3072-bit Discrete Log               & not PQC        & 384 B               & 32 B                 & 96 B               \\ \hline
    256-bit Elliptic Curve              & not PQC        & 32 B                & 32 B                 & 65 B               \\ \hline
    \end{tabular}
  }
  \caption{Confronto tra diversi algoritmi}
  \label{tab:confronto}
\end{table}

Una considerazione pratica sulla scelta tra gli algoritmi di crittografia post-quantistica è lo sforzo richiesto per inviare le chiavi pubbliche su Internet. Da questo punto di vista, gli algoritmi Ring-LWE, NTRU e SIDH forniscono chiavi di dimensioni comodamente inferiori a 1KB, le chiavi pubbliche con firma hash sono inferiori a 5KB e McEliece basato su MDPC richiede circa 1KB. D'altra parte, gli schemi Rainbow richiedono circa 125KB e McEliece basato su Goppa richiede una chiave di quasi 1MB.

Nel nostro caso di studio prenderemo in esame l'algoritmo basato su hash \textbf{SPHINCS}, che sembra essere un ottimo compromesso tra facilità d'utilizzo, supportabilità e dimensioni delle chiavi.

\section{Le vulnerabilità della blockchain nell'era quantistica}
Vediamo quindi ora quali sono i principali algoritmi quantistici che minacciano le attuali implementazioni della blockchain. In particolare, presenteremo l'algoritmo di Shor e di Grover, i potenziali rischi che questi comportano alle primitive crittografiche utilizzate dalla Blockchain.

\subsection{L'algoritmo di fattorizzazione di Shor}
Nel 1994, l'informatico teorico statunitense Peter Shor progetta un efficiente algoritmo quantistico capace di risolvere il problema della fattorizzazione di interi molto grandi in tempo polinomiale e non più in tempo esponenziale.

\begin{figure}[h]
  \centering
  \includegraphics[width=0.7\textwidth]{shor_example.png}
  \caption{Esempio di algoritmo di Shor per fattorizzare il numero 15}
  \label{fig:shor_example}
\end{figure}

L'algoritmo di Shor si basa sulla teoria per la fattorizzazione dei numeri.

Supponiamo di voler fattorizzare un numero \(N\), l'algoritmo:

\begin{enumerate}
  \item Controlla se \(N\) è numero primo o potenza di un numero primo, attraverso l'utilizzo di un qualsiasi test di primalità che sia polinomiale, e se è così si ferma, altrimenti passa al punto numero 2;
  \item Sceglie un numero casuale \(a\) tale che \(1 < a < N\);
  \item Se \(b = mcd\left(a, N\right) > 1\), dove \(mcd\) può essere calcolato in tempo polinomiale utilizzando l'algoritmo di Euclide, restituisce \(b\) e si ferma, altrimenti passa al punto numero 4;
  \item Trova l'ordine \(a \% N\) tale che
    \[
      a^r \equiv 1 \% N \;\; con \; r > 0
    \]
  \item Se \(r\) è dispari torna al punto numero 2, altrimenti passa al punto numero 6;
  \item Calcola
  \[
    x = a^{\frac{r}{2}} + 1 \% N
  \]
  \[
    y = a^{\frac{r}{2}} - 1 \% N
  \]
  \item Se \(x = 0\), torna al punto numero 2;
  \item Se \(y = 0\), prende \(r=\frac{r}{2}\) e torna al punto numero 5;
  \item Calcola \(p = gcd(x,N)\) e \(q = gcd(y,N)\). Uno tra i due sarà fattore non banale di \(N\).
\end{enumerate}

L'algoritmo appena illustrato potrebbe essere svolto in tempo ottimale anche da un computer classico se non fosse per il punto 4 che è computazionalmente molto oneroso, quindi l'ideale è utilizzare un computer quantistico. In termini di tempo, l'algoritmo di Shor può fattorizzare un intero \(N\) in tempo \(O(\log^3 N)\) e in spazio \(O(\log N)\).

\subsubsection{Algoritmo di Shor e minacce sulla Blockchain}
La maggior parte dei sistemi crittografici a chiave pubblica possono essere rotti utilizzando questo algoritmo quantistico, che andrà semplicemente ad utilizzare un numero di qubit pari al doppio della dimensione della chiave. Per comprendere al meglio il problema, basta prendere in considerazione l'RSA 2048: un computer classico con una CPU da 5 Ghz impiegherebbe circa 13,7 miliardi di anni per decifrarne un codice mentre un computer quantistico con CPU da 10 Mhz sarebbe in grado di fare ciò in circa 42 minuti\cite{kearney2021vulnerability}.

Lo schema di crittografia asimmetrico Rivest Shamir Adleman (RSA), che consiste nello scambio di messaggi tramite utilizzo di una chiave pubblica, che li cifra, e una privata, che li decifra, è molto simile al metodo utilizzato dalle tecnologie Blockchain per la creazione e la crittografia di wallet di criptovalute. In questo caso, quindi, viene generata una coppia di chiavi: quella pubblica, utilizzata per ricevere criptovalute e consultare il saldo presente sulla Blockchain, e quella privata, utilizzata per spendere le criptovalute. Questo tipo di schema si basa quindi su funzioni matematiche \textit{one-way} e numeri primi, motivo per il quale, l'applicazione dell'algoritmo quantistico di Shor, porterebbe alla violazione della crittografia RSA, con chiave a 2048 bit, attraverso l'utilizzo di un computer quantistico a 4096 qubit logici.

\subsection{L'algoritmo di ricerca di Grover}
Ideato nel 1996 da Lov Grover, è un algoritmo di ricerca che, sfruttando l'amplificazione d'ampiezza, è in grado di cercare un elemento o un valore, in un insieme non ordinato, in tempo \(O\left(\sqrt{N}\right)\) a differenza degli algoritmi classici che risolvono lo stesso problema in tempo \(O\left(N\right)\).

\begin{figure}[h]
  \centering
  \includegraphics[width=0.7\textwidth]{grover_example.png}
  \caption{Esempio di algoritmo di Glover per 3 qubit}
  \label{fig:grover_example}
\end{figure}

\subsubsection{Algoritmo di Grover e minacce sulla Blockchain}
L'algoritmo di consenso della Blockchain, si basa sul calcolo di funzioni
crittografiche hash che, a partire da un input, genera una stringa di byte a lunghezza fissa. La produzione di transizioni hash, però, renderebbe la Blockchain sicura e non manomettibile se non fosse che, l'algoritmo di Grover permette di individuare, con poco sforzo computazionele, i dati originali su cui è stato applicato l'hash: ciò permette la generazione di collisioni hash più efficiente rispetto alla ricerca a forza bruta, che richiede invece tempo lineare.

Inoltre, c'è da sottolineare che, nonostante gli attacchi tramite algoritmo di Grover sono considerati meno rischiosi rispetto a quelli di Shor, non è noto un sistema PoW abbastanza resistente a tali attacchi mentre, nel secondo caso, è possibile sostituire la crittografia vulnerabile con una crittografia post-quantistica, permettendo quindi di affrontare al meglio le minacce ricevute.

\chapter{QRChain}
Lo scopo di tale capitolo è la progettazione e l'implementazione di una blockchain decentralizzata basata sull'algoritmo di consenso PoS che chiameremo \textit{QRChain}, alla cui base vengono utilizzati non più algoritmi vulnerabili agli attacchi quantistici ma bensì algoritmi resistenti agli attacchi quantistici, nel nostro caso lo SPHINCS.

\section{Cos'è QRChain?}
Quantum Resistant Chain o, in breve, QRChain, inizialmente ideata con il nome di GoodChain \cite{Ghorbanzadeh_GoodChain_2022}, è una blockchain decentralizzata basata su Proof-of-Stake, scritta con Node.js.

All'interno di QRChain chiunque può essere un miner o un validatore. Ha due monete native: \textit{GTC} e \textit{MCT}. GTC è utilizzata per pagare le commissioni di transazione e MCT è utilizzata per pagare la convalida di un blocco. Gli MCT possono essere guadagnati puntando GTC.

Ogni volta che un convalidatore estrae un blocco, riceve una parte di GTC come ricompensa del blocco più le commissioni di transazione.
I validatori consumano anche alcuni MCT come tassa di estrazione. La quantità di MCT consumati è una percentuale della quantità totale di MCT che il validatore possiede.
Questo approccio offre maggiori possibilità di ottenere una ricompensa ai minatori con un numero inferiore di puntate. Inoltre, rende la rete più distribuita.

Inoltre, QRChain, introduce un meccanismo di "beneficenza" in cui ogni volta che un validatore estrae un blocco, il 20-25\% della rimcompensa del blocco, più le commissioni della transazione, viene devoluto a enti di beneficenza e ad altre organizzazioni simili.

\subsection{Meccanismo di consenso}
Per consenso si intende che la maggioranza dei validatori si è accordata sullo stesso blocco.
QRChain utilizza una versione personalizzata dell'algoritmo Proof-of-Stake per raggiungere questo obiettivo. Seleziona semplicemente il blocco candidato dal validatore con il maggior numero di MCT (non GCT).
La rete è tenuta al sicuro dal fatto che i nodi maligni devono avere costantemente il 51\% della quantità totale di MCT nei loro conti. Voglio ricordare che la tassa di estrazione è una percentuale della quantità totale di MCT che il validatore possiede. Quindi avere costantemente il 51\% dell'MCT totale è quasi impossibile. Quindi, già a livello di design, iniziamo a raggiungere un buon livello di sicurezza.

\subsection{Meccanismo di selezione della catena}
Nelle blockchain PoW come Bitcoin, la catena fiduciaria (o corretta) è la catena più lunga, che è determinata dalla difficoltà cumulativa totale della catena di Proof-of-Work. In altre parole, la catena che ha richiesto più energia per essere costruita.
Nelle blockchain PoS, come abbiamo detto, non viene richiesto un lavoro oneroso della CPU. Quindi è necessario un altro approccio per determinare la catena di fiducia.

In QRChain, ogni nodo elabora il blocco successivo e lo aggiunge al suo elenco di blocchi candidati. Poi inizia a ricevere l'elenco dei blocchi candidati degli altri nodi e li aggiunge al proprio elenco di blocchi candidati.
Quindi ogni nodo seleziona il blocco candidato con il maggior numero di MCT e aggiorna il proprio stato e la catena.
Poi ogni nodo cerca di assicurarsi che la sua catena sia quella corretta chiedendo agli altri nodi il loro ultimo blocco, se questo combacia con il proprio allora va tutto bene, altrimenti deve aggiornare o sostituire la sua catena.

Per questo, QRChain, utilizza un semplice meccanismo di reputazione. Il nodo da validare sceglie la catena tra i nodi di cui si fida di più. Questo elenco può essere inserito manualmente dal validatore o può utilizzare l'algoritmo predefinito per creare l'elenco.

L'algoritmo predefinito esamina la catena del nodo da validare e calcola un punto di fiducia per ogni validatore in base al numero di volte in cui il validatore ha estratto un blocco.
Quindi il nodo sceglierà la catena con il punto di fiducia più alto.

\section{Implementazione}
Vediamo nel dettaglio quali sono le componenti caratteristiche della libreria. La sostituzione degli algoritmi ha impattato la quasi totalità del codice poichè la libreria che implementa SPHINCS utilizza delle funzioni asincrone, quindi c'è stata la necessità di adattare le funzioni esistenti da funzioni asincrone a funzioni sincrone.

Invece, le funzioni che hanno il compito di firmare, validare e generare le chiavi sono principalmente 5:

\subsubsection{signBlock}
La funzione \textit{signBlock}, con parametro il blocco da firmare \textit{block}, firma un blocco con la \textit{privateKey} del validatore per assicurarsi che l'indirizzo appartenga a quest ultimo; un validatore non dovrebbe essere in grado di inserire indirizzi di altri, perchè in questo modo gli verrebbero sottratti alcuni MCT. Si basa sull'algoritmo illustrato nel paragrafo \ref{sec:sign_algorithm}.
\\
\begin{lstlisting}[language=JavaScript,breaklines]
  async signBlock(block) {
    const self = this
    const shaHash = self.hash(block)

    return await sphincs.sign(Buffer.from(shaHash, "utf8"), Buffer.from(self.validator.privateKey, 'hex'))
  }
\end{lstlisting}

\subsubsection{checkBlockSign}
La funzione \textit{checkBlockSign}, con parametro blocco da validare \textit{block}, firma \textit{sign} e chiave pubblica \textit{publicKey}, verifica se il blocco è stato firmato da chi sta validando o meno. Si basa sull'algoritmo illustrato nel paragrafo \ref{sec:check_algorithm}.
\\
\begin{lstlisting}[language=JavaScript,breaklines]
  async checkBlockSign(block, sign, publicKey) {
    const self = this
    const shaHash = self.hash(block)
    const verified = await sphincs.open(Buffer.from(sign, "base64"), Buffer.from(publicKey, 'hex'));

    return verified.toString() == shaHash
  }
\end{lstlisting}

\subsubsection{signTransaction}
La funzione \textit{signTransaction}, con parametro la transazione da firmare \textit{trx} e la chiave privata \textit{privateKey}, firma una transazione con la \textit{privateKey} del validatore. Si basa sull'algoritmo illustrato nel paragrafo \ref{sec:sign_algorithm}.
\\
\begin{lstlisting}[language=JavaScript,breaklines]
  async signTransaction(trx, privateKey) {
    const self = this;
    trx.from = QRChain.hex(trx.from);
    trx.to = QRChain.hex(trx.to);
    const trxHash = self.hash(trx);

    return await sphincs.sign(Buffer.from(trxHash, "utf8"), Buffer.from(privateKey, 'hex'))
  }
\end{lstlisting}

\subsubsection{checkTransactionSign}
La funzione \textit{checkTransactionSign}, con parametro transazione da validare \textit{trx}, firma \textit{sign} e chiave pubblica \textit{publicKey}, verifica se la transazione è stato firmato da chi sta validando o meno. Si basa sull'algoritmo illustrato nel paragrafo \ref{sec:check_algorithm}.
\\
\begin{lstlisting}[language=JavaScript,breaklines]
  async checkTransactionSign(trx, sign, publicKey) {
    const self = this;
    const trxHash = self.hash(trx);

    const verified = await sphincs.open(Buffer.from(sign, "base64"), Buffer.from(publicKey, 'hex'));
    return verified == trxHash;
  }
\end{lstlisting}

\subsubsection{generateKeyPairs}
La funzione \textit{generateKeyPairs}, con parametro la PATH dove memorizzare le chiavi, genera una coppia composta da chiave pubblica e privata utilizzando l'algoritmo SPHINCS. Si basa sull'algoritmo illustrato nel paragrafo \ref{sec:sec:key_generation}.
\\
\begin{lstlisting}[language=JavaScript,breaklines]
  static async generateKeyPairs(path) {
    const keyPair = await sphincs.keyPair();

    if (path) {
      mkdirp.sync(path);
      fs.writeFileSync(join(path, "public_key.pem"), Buffer.from(keyPair.publicKey).toString('hex'));
      fs.writeFileSync(join(path, "private_key.pem"), Buffer.from(keyPair.privateKey).toString('hex'));
      fs.writeFileSync(join(path, "public_key.hex"), Buffer.from(keyPair.publicKey).toString('hex'));
    }
    return keyPair;
  }
\end{lstlisting}

\subsubsection{Costruttore}
La classe principale, QRChain, è composta dalle seguenti variabili d'istanza e i seguenti parametri abbastanza autoesplicativi:
\begin{description}
  \item[chain] La lista di blocchi all'interno della catena.
  \item[state] Lo stato attuale della blockchain.
  \item[nodes] La lista contenente nodi, validatori o miner.
  \item[transactions\_pool] La lista delle transazioni correnti per il prossimo blocco candidato.
  \item[blockReward, validationFee] Sono rispettivamente la ricompensa del blocco per aver firmato minato un blocco (in GCT) e la percentuale di MCT che un validatore pagherà per validare un blocco. Entrambi impostati di default a \(30\) e \(0.25\).
  \item[feesToCharity, rewardToCharity, charityAddress] Sono rispettivamente la percentuale di tassa inviata in beneficenza, la percentuale della ricompensa del blocco che verrà inviata in beneficenza e l'indirizzo del wallet di beneficenza.
  \item[dbPath] È il percorso in cui avviene la persistenza dei dati. Nel dettaglio all'interno dei file \textit{state.json} e \textit{chain.json}. Questo parametro, se non viene specificato esplicitamente, corrisponderà alla PATH in cui viene eseguita la libreria.
\end{description}

Inoltre, viene inizializzato il validatore, la catena e il database di stati. Infine, avviene la creazione del blocco genesi.

\begin{lstlisting}[language=JavaScript,breaklines]
  ...
  class QRChain {
    constructor({ chain, state, nodes, dbPath, validator } = {}) {
      const self = this
      self.chain = chain || []
      self.state = state || {}
      self.nodes = nodes || [] 
      self.transactions_pool = []

      self.blockReward = 30
      self.validationFee = 0.25

      self.feesToCharity = 0.25
      self.rewardToCharity = 0.2
      self.charityAddress = "..." 

      self.executeScriptPath = dirname(process.argv[1])
      self.dbPath = dbPath || self.executeScriptPath 

      return (async () => {
        self.validator = await self.init_validator(validator)
        await self.init_database()
        if (self.chain.length == 0) {
          await self.genesis_block()
        }
        return self
      })();
    }
  ...
\end{lstlisting}

\subsection{Tecnologie usate}
\subsubsection{JavaScript}
QRChain è stata implementata completamente in JavaScript. Quest'ultimo, a volta abbreviato con JS, è un linguaggio di programmazione multi paradigma orientato agli eventi, standardizzato per la prima volta il 1997 dalla ECMA con il nome ECMAScript, l'ultimo standard, di giugno 2022, è ECMA-262 Edition 13 ed è anche uno standard ISO (ISO/IEC 16262).

\subsubsection{NodeJs}
Node.js è un runtime system open source multipiattaforma orientato agli eventi per l'esecuzione di codice JavaScript, costruito sul motore JavaScript V8 di Google Chrome.

\chapter{Conclusioni e sviluppi futuri}
IMPLEMENTARE HARD FORK DI BLOCKCHAIN NOTE MA CON POS-QR.
%%%%%%%%%%%%%%%%%%%%%%%%%%%%
\printbibliography

\end{document}
