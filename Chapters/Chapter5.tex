\chapter{Conclusioni e sviluppi futuri}
In questo lavoro di tesi è stata presentata QRChain, un modello di blockchain basata su Proof-of-Stake, con lo scopo di introdurre problemi e soluzioni delle attuali blockchain in vista del sempre più rapido sviluppo dei computer quantistici. Come primo passo sono state analizzate le vulnerabilità note delle blockchain odierne. La prima vulnerabilità riguarda gli attacchi effettuati tramite l'algoritmo di ricerca di Grover e l'algoritmo di fattorizzazione di Shor. Successivamente, sono state fatte le considerazioni sulla progettazione del sistema blockchain. La soluzione adottata si basa sulla sostituzione dell'attuale schema di firma digitale sostituendo il vecchio, non quantum-safe, con uno sicuro. La scelta è ricaduta sullo SPHINCS che risulta essere quantum-safe.

Infine, negli sviluppi futuri, è stato proposta l'implementazione di una \textit{hard fork} di Ethereum ma con alla base QRChain.