\chapter{QRChain}
Lo scopo di tale capitolo è la progettazione e l'implementazione di una blockchain decentralizzata basata sull'algoritmo di consenso PoS che chiameremo \textit{QRChain}, alla cui base vengono utilizzati non più algoritmi vulnerabili agli attacchi quantistici ma bensì algoritmi resistenti agli attacchi quantistici, nel nostro caso lo SPHINCS.

\section{Cos'è QRChain?}
Quantum Resistant Chain o, in breve, QRChain, inizialmente ideata con il nome di GoodChain \cite{Ghorbanzadeh_GoodChain_2022}, è una blockchain decentralizzata basata su Proof-of-Stake, scritta con Node.js.

All'interno di QRChain chiunque può essere un miner o un validatore. Ha due monete native: \textit{GTC} e \textit{MCT} \ref{tab:gtc_mct}. GTC è utilizzata per pagare le commissioni di transazione e MCT è utilizzata per pagare la convalida di un blocco. Gli MCT possono essere guadagnati puntando GTC.

% Please add the following required packages to your document preamble:
% \usepackage{graphicx}
% Please add the following required packages to your document preamble:
% \usepackage{graphicx}
\begin{table}[]
  \resizebox{\columnwidth}{!}{%
  \begin{tabular}{|ll|}
  \hline
  \multicolumn{1}{|c|}{\textbf{GTC}} & \multicolumn{1}{c|}{\textbf{MCT}} \\ \hline
  \multicolumn{2}{|l|}{\textbf{Scopo}} \\ \hline
  \multicolumn{1}{|l|}{• Pagare le commissioni di transazione} & \begin{tabular}[c]{@{}l@{}}• Validazione di un blocco\\ • Tassa di mining\end{tabular} \\ \hline
  \multicolumn{2}{|l|}{\textbf{Ottenimento}} \\ \hline
  \multicolumn{1}{|l|}{• Mining di un blocco} & • Mettendo in gioco (staking) GTC \\ \hline
  \end{tabular}%
  }
  \caption{GTC e MCT}
  \label{tab:gtc_mct}
\end{table}

Ogni volta che un convalidatore estrae un blocco, riceve una parte di GTC come ricompensa del blocco più le commissioni di transazione.
I validatori consumano anche alcuni MCT come tassa di estrazione. La quantità di MCT consumati è una percentuale della quantità totale di MCT che il validatore possiede.
Questo approccio offre maggiori possibilità di ottenere una ricompensa ai minatori con un numero inferiore di puntate. Inoltre, rende la rete più distribuita.

Inoltre, QRChain, introduce un meccanismo di "beneficenza" in cui ogni volta che un validatore estrae un blocco, il 20-25\% della rimcompensa del blocco, più le commissioni della transazione, viene devoluto a enti di beneficenza e ad altre organizzazioni simili.

\subsection{Meccanismo di consenso}
Per consenso si intende che la maggioranza dei validatori si è accordata sullo stesso blocco.
QRChain utilizza una versione personalizzata dell'algoritmo Proof-of-Stake per raggiungere questo obiettivo. Seleziona semplicemente il blocco candidato dal validatore con il maggior numero di MCT (non GCT).
La rete è tenuta al sicuro dal fatto che i nodi maligni devono avere costantemente il 51\% della quantità totale di MCT nei loro conti. Voglio ricordare che la tassa di estrazione è una percentuale della quantità totale di MCT che il validatore possiede. Quindi avere costantemente il 51\% dell'MCT totale è quasi impossibile. Quindi, già a livello di design, iniziamo a raggiungere un buon livello di sicurezza.

\subsection{Meccanismo di selezione della catena}
Nelle blockchain PoW come Bitcoin, la catena fiduciaria (o corretta) è la catena più lunga, che è determinata dalla difficoltà cumulativa totale della catena di Proof-of-Work. In altre parole, la catena che ha richiesto più energia per essere costruita.
Nelle blockchain PoS, come abbiamo detto, non viene richiesto un lavoro oneroso della CPU. Quindi è necessario un altro approccio per determinare la catena di fiducia.

In QRChain, ogni nodo elabora il blocco successivo e lo aggiunge al suo elenco di blocchi candidati. Poi inizia a ricevere l'elenco dei blocchi candidati degli altri nodi e li aggiunge al proprio elenco di blocchi candidati.
Quindi ogni nodo seleziona il blocco candidato con il maggior numero di MCT e aggiorna il proprio stato e la catena.
Poi ogni nodo cerca di assicurarsi che la sua catena sia quella corretta chiedendo agli altri nodi il loro ultimo blocco, se questo combacia con il proprio allora va tutto bene, altrimenti deve aggiornare o sostituire la sua catena.

Per questo, QRChain, utilizza un semplice meccanismo di reputazione. Il nodo da validare sceglie la catena tra i nodi di cui si fida di più. Questo elenco può essere inserito manualmente dal validatore o può utilizzare l'algoritmo predefinito per creare l'elenco.

L'algoritmo predefinito esamina la catena del nodo da validare e calcola un punto di fiducia per ogni validatore in base al numero di volte in cui il validatore ha estratto un blocco.
Quindi il nodo sceglierà la catena con il punto di fiducia più alto.

\section{Implementazione}
Vediamo nel dettaglio quali sono le componenti chiave della libreria sviluppata. La sostituzione degli algoritmi ha impattato la quasi totalità del codice poichè la libreria che implementa SPHINCS utilizza delle funzioni asincrone, quindi c'è stata la necessità di adattare le funzioni esistenti da funzioni asincrone a funzioni sincrone.

Invece, le funzioni che hanno il compito di firmare, validare e generare le chiavi sono principalmente 5:

\subsubsection{signBlock}
La funzione \textit{signBlock}, con parametro il blocco da firmare \textit{block}, firma un blocco con la \textit{privateKey} del validatore per assicurarsi che l'indirizzo appartenga a quest ultimo; un validatore non dovrebbe essere in grado di inserire indirizzi di altri, perchè in questo modo gli verrebbero sottratti alcuni MCT. Si basa sull'algoritmo illustrato nel paragrafo \ref{sec:sign_algorithm}.
\\
\begin{lstlisting}[language=JavaScript,breaklines]
  async signBlock(block) {
    const self = this
    const shaHash = self.hash(block)

    return await sphincs.sign(Buffer.from(shaHash, "utf8"), Buffer.from(self.validator.privateKey, 'hex'))
  }
\end{lstlisting}

\subsubsection{checkBlockSign}
La funzione \textit{checkBlockSign}, con parametro il blocco da validare \textit{block}, la firma \textit{sign} e la chiave pubblica \textit{publicKey}, verifica se il blocco è stato firmato da chi sta validando o meno. Si basa sull'algoritmo illustrato nel paragrafo \ref{sec:check_algorithm}.
\\
\begin{lstlisting}[language=JavaScript,breaklines]
  async checkBlockSign(block, sign, publicKey) {
    const self = this
    const shaHash = self.hash(block)
    const verified = await sphincs.open(Buffer.from(sign, "base64"), Buffer.from(publicKey, 'hex'));

    return verified.toString() == shaHash
  }
\end{lstlisting}

\subsubsection{signTransaction}
La funzione \textit{signTransaction}, con parametro la transazione da firmare \textit{trx} e la chiave privata \textit{privateKey}, firma una transazione con la \textit{privateKey} del validatore. Si basa sull'algoritmo illustrato nel paragrafo \ref{sec:sign_algorithm}.
\\
\begin{lstlisting}[language=JavaScript,breaklines]
  async signTransaction(trx, privateKey) {
    const self = this;
    trx.from = QRChain.hex(trx.from);
    trx.to = QRChain.hex(trx.to);
    const trxHash = self.hash(trx);

    return await sphincs.sign(Buffer.from(trxHash, "utf8"), Buffer.from(privateKey, 'hex'))
  }
\end{lstlisting}

\subsubsection{checkTransactionSign}
La funzione \textit{checkTransactionSign}, con parametro la transazione da validare \textit{trx}, la firma \textit{sign} e la chiave pubblica \textit{publicKey}, verifica se la transazione è stata firmata da chi sta validando o meno. Si basa sull'algoritmo illustrato nel paragrafo \ref{sec:check_algorithm}.
\\
\begin{lstlisting}[language=JavaScript,breaklines]
  async checkTransactionSign(trx, sign, publicKey) {
    const self = this;
    const trxHash = self.hash(trx);

    const verified = await sphincs.open(Buffer.from(sign, "base64"), Buffer.from(publicKey, 'hex'));
    return verified == trxHash;
  }
\end{lstlisting}

\subsubsection{generateKeyPairs}
La funzione \textit{generateKeyPairs}, con parametro la PATH dove memorizzare le chiavi, genera una coppia composta da chiave pubblica e privata utilizzando l'algoritmo SPHINCS. Si basa sull'algoritmo illustrato nel paragrafo \ref{sec:key_generation}.
\\
\begin{lstlisting}[language=JavaScript,breaklines]
  static async generateKeyPairs(path) {
    const keyPair = await sphincs.keyPair();

    if (path) {
      mkdirp.sync(path);
      fs.writeFileSync(join(path, "public_key.pem"), Buffer.from(keyPair.publicKey).toString('hex'));
      fs.writeFileSync(join(path, "private_key.pem"), Buffer.from(keyPair.privateKey).toString('hex'));
      fs.writeFileSync(join(path, "public_key.hex"), Buffer.from(keyPair.publicKey).toString('hex'));
    }
    return keyPair;
  }
\end{lstlisting}

\subsubsection{Costruttore}
La classe principale, QRChain, è composta dalle seguenti variabili d'istanza e i seguenti parametri abbastanza autoesplicativi:
\begin{description}
  \item[chain] La lista di blocchi all'interno della catena.
  \item[state] Lo stato attuale della blockchain.
  \item[nodes] La lista contenente nodi, validatori o miner.
  \item[transactions\_pool] La lista delle transazioni correnti per il prossimo blocco candidato.
  \item[blockReward, validationFee] Sono rispettivamente la ricompensa del blocco per aver firmato minato un blocco (in GCT) e la percentuale di MCT che un validatore pagherà per validare un blocco. Entrambi impostati di default a \(30\) e \(0.25\).
  \item[feesToCharity, rewardToCharity, charityAddress] Sono rispettivamente la percentuale di tassa inviata in beneficenza, la percentuale della ricompensa del blocco che verrà inviata in beneficenza e l'indirizzo del wallet di beneficenza.
  \item[dbPath] È il percorso in cui avviene la persistenza dei dati. Precisamente all'interno dei file \textit{state.json} e \textit{chain.json}. Questo parametro, se non viene specificato esplicitamente, corrisponderà alla PATH in cui viene eseguita la libreria.
\end{description}

Inoltre, viene inizializzato il validatore, la catena e il database di stati. Infine, avviene la creazione del blocco genesi.

\begin{lstlisting}[language=JavaScript,breaklines]
  ...
  class QRChain {
    constructor({ chain, state, nodes, dbPath, validator } = {}) {
      const self = this
      self.chain = chain || []
      self.state = state || {}
      self.nodes = nodes || [] 
      self.transactions_pool = []

      self.blockReward = 30
      self.validationFee = 0.25

      self.feesToCharity = 0.25
      self.rewardToCharity = 0.2
      self.charityAddress = "..." 

      self.executeScriptPath = dirname(process.argv[1])
      self.dbPath = dbPath || self.executeScriptPath 

      return (async () => {
        self.validator = await self.init_validator(validator)
        await self.init_database()
        if (self.chain.length == 0) {
          await self.genesis_block()
        }
        return self
      })();
    }
  ...
\end{lstlisting}

\subsection{Tecnologie usate}
\subsubsection{JavaScript}
QRChain è stata implementata completamente in JavaScript, abbreviato con JS, è un linguaggio di programmazione multi paradigma orientato agli eventi, standardizzato per la prima volta il 1997 dalla ECMA con il nome ECMAScript, l'ultimo standard, di giugno 2022, è ECMA-262 Edition 13 ed è anche uno standard ISO (ISO/IEC 16262).

\subsubsection{NodeJs}
Node.js è un runtime system open source multipiattaforma orientato agli eventi per l'esecuzione di codice JavaScript, costruito sul motore JavaScript V8 di Google Chrome.