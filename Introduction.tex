\chapter*{Introduzione}
\addcontentsline{toc}{chapter}{\protect\numberline{}Introduzione}
Possiamo sicuramente affermare che dalla pubblicazione del white paper \textit{"Bitcoin: A Peer-to-Peer Eletronic Cash System"}, pubblicato nel 2008 da Satoshi Nakamoto, il mondo digitale sta subendo un drastico cambiamento. In molti hanno iniziato ad investire in questa nuove monete digitali figlie di questa nuova tecnologia, chiamata \textbf{Blockchain}. Oltre ad avver dato vita a questa nuova forma monetaria, la Blockchain si sta spingendo verso nuovi orizzonti d'utilizzo, come ad esempio la pubblica amministrazione e la sanità. Una delle principali caratteristiche della Blockchain è la mancanza di un'entità centrale.

Ma ad oggi sta emergendo una nuova tecnologia che va a compromettere i punti di forza della blockchain, il \textbf{quantum computing}. Quest'ultima tecnologia è senza dubbio in rapida crescita, passando da pochi qbit (l'unità di informazione quantistica) a centinaia di qbit in soli cinque anni, e si prevede un'ulteriore crescita fino al 2023 toccando numeri a quattro cifre. Tutta questa capacità computazionale data dai computer quantistici va a minacciare la robustezza degli attuali algoritmi di firma digitale che sono alla base della blockchain. Quindi bisogna trovare al più presto una soluzione che renda gli attuali schemi di firma digitale resistenti agli attacchi quantistici.

È stata proposta una variante di \textbf{Proof-of-Stake} a cui viene rimpiazzato l'attuale algoritmo di cifratura con uno schema quantum safe, lo \textit{SPHINCS}. Inoltre, è stata presentata un'implementazione del sistema proposto, \textbf{Quantum Resistant Chain} (o \textbf{QRChain}).

Infine, negli sviluppi futuri, è stato proposto un aggiornamento di Ethereum in cui si sostituisce l'attuale algoritmo di consenso, la \textbf{Proof-of-Work}, con la più sicura Proof-of-Stake predisponendo quindi Ethereum ad accogliere QRChain nel proprio core. Oltre all'aggiornamento di Ethereum è stato proposto l'aggiornamento dello schema di cifratura SPHINCS con il più recente \textit{SPHINCS+} che porta notevoli miglioramenti sia dal punto di vista delle performance che della sicurezza.

L'elaborato si articola su cinque capitoli:
\begin{itemize}
  \item Nel \textbf{Capitolo 1} vengono affrontati nello specifico i fondamenti della computazione quantistica. Dopodichè si forniscono i metodi di gestione di quest'ultima e come avviene la codifica dell'informazione reale all'interno di un sistema quantistico.
  \item Nel \textbf{Capitolo 2} viene illustrata la storia della Blockchain, partendo dai Sistemi di Chaum fino ad arrivare al Bitcoin. Viene poi illustrata l'architettura e il suo funzionamento, i diversi tipi di algoritmi di consenso e le generazioni di blockchain fino ad ora.
  \item Nel \textbf{Capitolo 3} vengono illustrate le principali criticità della Proof-of-Stake, i modelli di attacco all'algoritmo e come queste debolezze possono essere mitigate grazie ad accorgimenti architetturali e la sostituzione degli schemi di firma digitale. Viene infine presentato lo SPHINCS e i suoi punti di forza.
  \item Nel \textbf{Capitolo 4} viene presentata QRChain, il relativo meccanismo di consenso, il meccanismo di selezione della catena e l'implementazione di quest'ultima. QRChain è l'implementazione di una blockchain avente come algoritmo di consenso la Proof-of-Stake e come schema di crifratura lo SPHINCS.
  \item Nel \textbf{Capitolo 5} viene proposto un aggiornamento ad Ethereum basato su Proof-of-Stake e metodi di cifratura quantum safe, chiamata Ethereum 2.0, e il successore dello SPHINCS avente diverse migliorie computazionali, chiamato SPHINCS+. 
\end{itemize}
