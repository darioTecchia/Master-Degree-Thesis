\chapter*{Introduzione}
\addcontentsline{toc}{chapter}{\protect\numberline{}Introduzione}
Possiamo sicuramente affermare che dalla pubblicazione del white paper \textit{"Bitcoin: A Peer-to-Peer Eletronic Cash System"}, pubblicato nel 2008 da Satoshi Nakamoto, il mondo informatico e reale sta subendo un drastico cambiamento dando un nuovo significato al concetto di ricchezza. In molti hanno iniziato ad investire in questa nuove monete digitali figlie di questa nuova tecnologia, chiamata Blockchain. Oltre ad avver dato vita a questa nuova forma monetaria, la Blockchain si sta ampliando verso nuovi contesti d'utilizzo, come ad esempio la pubblica amministrazione, la sanità, ecc. Una delle principali caratteristiche della Blockchain è la mancanza di un'entità centrale, come ad esempio la banca per quanto riguarda il concetto di ricchezza.

Ma ad oggi sta emergendo una nuova tecnologia che va a compromettere i punti di forza della blockchain, il quantum computing. Quest'ultima tecnologia è senza dubbio in rapida crescita, passando da pochi qbit () a centinaia di qbit in soli cinque anni, e si prevede un ulteriore crescita fino al 2023 toccando numeri a quattro cifre. Tutta questa capacità computazionale data dai computer quantistici va a minacciare la robustezza degli attuali algoritmi di firma digitale che sono alla base della blockchain. Quindi bisogna trovare al più presto una soluzione che renda gli attuali schemi di firma digitale resistenti agli attacchi quantistici.

È stato proposta una variante di Proof-of-Stake alla cui base vengono rimpiazzati gli attuali algoritmi di cifratura con schemi quantum safe, lo SPHINCS. Inoltre, è stata presentata un'implementazione di quest'ultima, Quantum Resistant Chain (o QRChain).

Infine, negli sviluppi futuri, è stato proposto un aggiornamento di Ethereum in cui si sostituisce l'attuale algoritmo di consenso, la Proof-of-Work, con la più sicura Proof-of-Stake predisponendo quindi Ethereum ad accogliere QRChain nel proprio core. Oltre all'aggiornamento di Ethereum è stato proposto l'aggiornamento dello schema di cifratura SPHINCS con il più recente SPHINCS+ che porta notevoli miglioramenti sia dal punto di vista della performance che della sicurezza.

L'elaborato si articola su cinque capitoli:
\begin{description}
  \item[Capitolo 1] Vengono affrontati nello specifico i fondamenti della computazione quantistica. Dopodichè, si forniscono i metodi di gestione di quest'ultima e come avviene la codifica dell'informazione reale all'interno di un sistema quantistico.
  \item[Capitolo 2] Viene illustrata la storia della Blockchain, partendo dai sistemi di Chaum fino ad arrivare al Bitcoin. Viene poi illustrata l'architettura e il suo funzionamento, i diversi tipi di algoritmi di consenso e le generazioni di blockchain fino ad ora.
  \item[Capitolo 3] Vengono illustrate le principali criticità della Proof-of-Stake, i modelli di attacco all'algoritmo e come queste debolezze possono essere mitigate grazie ad accorgimenti architettonici e la sostituzione degli schemi di firma digitale. Viene infine presentato lo SPHINCS e i suoi punti di forza.
  \item[Capitolo 4] Viene presentata QRChain, il relativo meccanismo di consenso, il meccanismo di selezione della catena e l'implementazione di quest'ultima.
  \item[Capitolo 5] Viene proposto un aggiornamento ad Ethereum, Ethereum 2.0, e il successore dello SPHINCS, SPHINCS+. 
\end{description}
