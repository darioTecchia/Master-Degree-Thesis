\chapter*{Introduzione}
\addcontentsline{toc}{chapter}{\protect\numberline{}Introduzione}%
Si può asserire con certezza che il sistema internazionale economico ha subito un drastico cambiamento dopo l'introduzione di concetti come la \textit{blockchain}, il \textit{Bitcoin} e, più in generale, le \textit{criptovalute}. La rivoluzione è iniziata nel 2008 quando venne pubblicato il \textit{white paper} intitolato \textit{"Bitcoin: A Peer-to-Peer Eletronic Cash System"} di Satoshi Nakamoto \cite{bitcoin-white-paper}. Da allora, la blockchain ha subito rapidamente innumerevoli evoluzioni e ha visto ampliarsi notevolmente i suoi campi di utilizzo, passando dall'essere una semplice base per le monete  elettroniche, per poi essere utilizzata come fondamento per il concetto di  \textit{smart contract} e \textit{DApp}\footnote{App Decentralizzate simili alle app tradizionali, con la differenza fondamentale che al posto di appoggiarsi su server centralizzati sfruttano le piattaforme blockchain e il loro network distribuito.}, fino all'avvento degli \textit{NFT (Non Fungible Token)}.

Come vedremo, la blockchain porta un numero grande di vantaggi che però rischiano di essere compromessi con l'avvento di una recente tecnologia che sta prendendo sempre più piede, il \textit{quantum computing}. Il quantum computing o calcolo quantistico è una tecnologia emergente che sfrutta le leggi della meccanica quantistica per risolvere problemi troppo complessi per i computer classici. Analogamente all'informatica tradizionale, anche il calcolo quantistico ha un'unità base che è il \textit{qubit}. L'evoluzione di questi calcolatori, in contrario a quanto affermato dalla Legge di Moore, sta avvenendo in maniera esponenziale passando in pochi anni da pochi qbit ad un centinaio.

Tutta questa capacità computazionale può mettere in seria difficoltà gli attuali algoritmi di firma con l'\textit{ECDSA}\footnote{In crittografia, l'Elliptic Curve Digital Signature Algorithm offre una variante del Digital Signature Algorithm usando la crittografia ellittica.} che è alla base di bitcoin e di innumerevoli altri sistemi altamente critici. Bisogna quindi trovare un modo per rendere la blockchain immune a questa nuova potenza computazionale. L'implementazione e lo studio degli attuali metodi di protezione proposti sono alla base di questo lavoro di tesi.

È stato implementato un nuovo modello di blockchain, chiamato bitcoin QR, che nasce come fork dell'attuale bitcoin. L'algoritmo hashcash, alla base della proof of work di bitcoin classico, è stato sostituito con \textit{equiash}, che risulta essere resistente nei confronti degli attacchi quantistici basati sull'algoritmo di Grover. Dopodichè, una volta aver analizzato i vari algoritmi di firma post quantistica, l'algoritmo \textit{ECDSA}\footnotemark[2] è stato sostituito con XMSS che, come suggerito dal \textit{PQCRYPTO (European Consortium of Universities and Companies for Post-Quantum Cryptography Issues)}, risulta essere quantum safe. Infine, è stata proposta una possibile rete ausiliaria off-chain volta a risolvere i problemi di scalabilità di bitcoin, alternativa alla lighting network, denominata quantum lighting network, resa sicura tramite la distribuzione a chiave quantistica QKD.

L'elaborato di tesi si articola sui seguenti capitoli.
\begin{itemize}
  \item Nel \textbf{Capitolo 1} viene fatta un'introduzione all'elaborato.
  \item Nel \textbf{Capitolo 2} viene approfondito il concetto di computer-quantistica, da come avviene la rappresentazione dell'informazione fino alla gestione di quest'ultima.
  \item Nel \textbf{Capitolo 3} viene introdotto il concetto di blockchain, dalla nascita fino alle idee rivoluzionarie di Satoshi Nakamoto.
  \item Nel \textbf{Capitolo 4} vengono illustrate le debolezze della blockchain nell'era quantistica e come queste posso essere irrobustite.
  \item Nel \textbf{Capitolo 5} vengono implementati gli irrobustimenti introdotti nei capitoli precedenti.
  \item Nel \textbf{Capitolo 6} vengono tratte le conclusioni e i possibili sviluppi futuri.
\end{itemize}